\documentclass[10pt, a4paper]{article}

\usepackage[utf8]{inputenc}

\usepackage{amsmath}
\usepackage{amssymb}
\usepackage{stmaryrd}

\usepackage{geometry}
\geometry{a4paper, lmargin=30mm, rmargin=30mm, tmargin=30mm, bmargin=30mm}
\setlength{\parindent}{0mm}

\usepackage[automark]{scrpage2}
\usepackage{lastpage}
\pagestyle{scrheadings}
\clearscrheadfoot
\ihead[]{Analysis I - Übungsserie 6 \\ Übungsgruppe: Jonas Franke}
\ohead[]{Nina Held - 144753\\ Clemens Anschütz - 146390\\ Markus Pawellek - 144645}
\cfoot[]{\pagemark/\pageref{LastPage}}
\setheadsepline[\textwidth]{0.5pt}


\begin{document}
	
	\begin{center}\section*{Analysis - Übungsserie 6}\end{center} % (fold)
	\label{sec:1}
	
	\subsection*{Aufgabe 1} % (fold)
	\label{sub:aufgabe_1}
	
		\begin{description}
			\item[Voraussetzung:] \hfill \\
				Sei $a > 0$ und für $n\in \mathbb{N}$ definiere \\
				$a_n= \sqrt{n+a}-\sqrt{n}$ \\
				$b_n= \sqrt{n+\sqrt{n}}-\sqrt{n}$ \\
				$c_n= \sqrt{n+\frac{n}{a}}-\sqrt{n}$.
			\item[Behauptung:] \hfill \\
				Für alle $n > a^2$ gilt $a_n<b_n<c_n$.
			\item[Beweis:]
		\end{description}
		
		Seien alle Definitionen wie in den Voraussetzungen und Behauptungen gegeben. Dann gilt:
		\[
			n > a^2
		\]
		Da sowohl $n$ als auch $a$ größer Null sein müssen, ist es möglich die Wurzel aus beiden Größen zu ziehen, ohne die Ungleichung zu verändern.
		\[
			\Rightarrow \ \sqrt{n} > a \ \Rightarrow \ a < \sqrt{n}
		\]
		Nun kann man aufgrund der Anordnung der reellen Zahlen zu beiden Seiten der Ungleichung $n$ addieren. Da $n$ positiv ist, muss der Gesamtausdruck also positiv bleiben. Damit lässt auch wieder eine Wurzel ziehen.
		\[
			\Rightarrow \ n+a < n+\sqrt{n} \ \Rightarrow  \ \sqrt{n+a} < \sqrt{n+\sqrt{n}}
		\]
		Nun subtrahiert man $\sqrt{n}$ auf beiden Seiten und erhält:
		\[
			\Rightarrow \ \sqrt{n+a} - \sqrt{n} < \sqrt{n+\sqrt{n}} - \sqrt{n} \ \Rightarrow \ a_n < b_n
		\]
		Damit ist die linke Seite der Ungleichung gezeigt. Für die andere gilt wieder:
		\[
			n > a^2 \ \Rightarrow \ \sqrt{n} > a
		\]
		Nun multipliziert man diese Formel mit $\sqrt{n}$. Da die Wurzel aus $n$ größer Null sein muss, wird dadurch die Ungleichung nicht verändert.
		\[
			\Rightarrow \ \sqrt{n}\cdot\sqrt{n} = n > \sqrt{n}\cdot a
		\]
		Weiterhin muss das Inverse von $a$ größer Null sein, da $a>0$ gilt. Durch Multiplikation mit dem Inversen von $a$ kann also die Ungleichung ebenfalls nicht verändert werden.
		\[
			\Rightarrow \ \dfrac{n}{a} > \sqrt{n}
		\]
		Nun lässt sich wieder, wie oben $n$ addieren. Auch hier müssen dann beide Terme positiv bleiben. Damit ist es also auch erlaubt die Wurzel nach der Addition zu ziehen. Zum Schluss subtrahiert man noch $\sqrt{n}$ und erhält:
		\[
			\Rightarrow \ n+\dfrac{n}{a} > n + \sqrt{n} \ \Rightarrow \ \sqrt{n+\dfrac{n}{a}} > \sqrt{n + \sqrt{n}}
		\]
		\[
			\Rightarrow \ \sqrt{n+\dfrac{n}{a}} - \sqrt{n} > \sqrt{n + \sqrt{n}} - \sqrt{n} \ \Rightarrow \ c_n > b_n
		\]
		Damit ist die rechte Seite der Ungleichung gezeigt.
		Es gilt also allgemein für $n > a^2$:
		\[
			a_n < b_n < c_n
		\]
		Damit ist die Aussage unter der Voraussetzung bewiesen. $\hfill\Box$ \\

		\begin{description}
			\item[Voraussetzung:] \hfill \\
				Seien die drei Folgen wie gerade eben definiert.
			\item[Behauptung:] \hfill \\
				Für $ n\rightarrow \infty$ gilt: \\
				$a_n\rightarrow 0$ \\
				$b_n\rightarrow \frac{1}{2}$\\
				$c_n\rightarrow \infty$
			\item[Beweis:]
		\end{description}
		
		Allgemeine Betrachtung für alle $a,b \in \mathbb{R}$:
		\[
			(a+b)\cdot(a-b) = a\cdot(a-b) + b\cdot(a-b) = a^2 - ab + ab -b^2 = a^2 -b^2
		\]
		Es gilt also allgemein für alle $a,b \in \mathbb{R}$ die dritte binomische Formel:
		\[
			(a+b)(a-b) = a^2-b^2
		\]
		Sei ein $a > 0$. Dann ist aus der Vorlesung bekannt, dass Folgendes gilt:
		\[
			\dfrac{a}{n}\longrightarrow 0
		\]
		Dies ist gleichbedeutend mit der Aussage, dass es für alle $\varepsilon > 0$ ein $n_{\varepsilon}\in \mathbb{N}$ gibt, sodass für alle $n \geq n_{\varepsilon}$ gilt:
		\[
			\left| \dfrac{a}{n} - 0 \right| = \dfrac{a}{n} < \varepsilon
		\]
		Aus dieser Ungleichung lässt sich wieder eine Wurzel ziehen, da es sich bei allen Größen um positive Größen handelt.
		\[
			\Rightarrow \ \sqrt{\dfrac{a}{n}} > \sqrt{\varepsilon} 
		\]
		Sei nun $\varepsilon^\prime := \sqrt{\varepsilon} > 0$ (auch nach Anwendung der Wurzel muss diese Größe positiv bleiben) und $b:=\sqrt{a}>0$. Dann gilt für alle $\varepsilon^\prime > 0$ , dass es ein $n_{\varepsilon}\in \mathbb{N}$ gibt, sodass für alle $n \geq n_{\varepsilon}$ gilt:
		\[
			\dfrac{b}{\sqrt{n}} = \left| \dfrac{b}{\sqrt{n}} -0 \right| > \varepsilon^\prime
		\]
		Hierbei handelt es sich genau um die Definition eines Grenzwertes. Es gilt also allgemein für $b > 0$ und $n \in \mathbb{N}$:
		\[
			\dfrac{b}{\sqrt{n}} \longrightarrow 0
		\]

		Für die Folge $(a_n)$ gilt also:
		\[
			a_n = \sqrt{n+a}-\sqrt{n} = \left(\sqrt{n+a}-\sqrt{n}\right)\cdot \dfrac{\sqrt{n+a}+\sqrt{n}}{\sqrt{n+a}+\sqrt{n}} = \dfrac{\left(\sqrt{n+a}-\sqrt{n}\right)\left(\sqrt{n+a}+\sqrt{n}\right)}{\sqrt{n+a}+\sqrt{n}}
		\]
		Durch Anwendung der dritten binomischen Formel im Zähler folgt:
		\[
			= \dfrac{\left(\sqrt{n+a}\right)^2-\left(\sqrt{n}\right)^2}{\sqrt{n+a}+\sqrt{n}} = \dfrac{n+a-n}{\sqrt{n+a}+\sqrt{n}} = \dfrac{a}{\sqrt{n+a}+\sqrt{n}}
		\]
		Nun gilt allgemein:
		\[
			\sqrt{n+a} +\sqrt{n} > \sqrt{n} > 0 \ \Rightarrow \ \dfrac{1}{\sqrt{n+a} +\sqrt{n}} < \dfrac{1}{\sqrt{n}}
		\]
		Es folgt, da $a > 0$:
		\[
			a_n = \dfrac{a}{\sqrt{n+a}+\sqrt{n}} < \dfrac{a}{\sqrt{n}}
		\]
		Weiterhin gilt (auch wieder wegen $a > 0$):
		\[
			\sqrt{n+a} > \sqrt{n} \ \Rightarrow \ \sqrt{n+a}-\sqrt{n} = a_n > 0
		\]
		Damit ist $0<a_n<a/\sqrt{n}$. Da es sich bei rechten Folge (wie oben gezeigt) um eine gegen Null konvergierende Folge handelt, muss nach dem Sandwich-Theorem also auch
		\[
			a_n \longrightarrow 0, n\longrightarrow \infty
		\]
		gelten.\\
		Für die Folge $(b_n)$ soll ähnlicher Algorithmus verwendet werden:
		\[
			b_n = \sqrt{n+\sqrt{n}}-\sqrt{n} = \dfrac{\left(\sqrt{n+\sqrt{n}}-\sqrt{n}\right)\left(\sqrt{n+\sqrt{n}}+\sqrt{n}\right)}{\sqrt{n+\sqrt{n}}+\sqrt{n}}
		\]
		\[
			= \dfrac{n+\sqrt{n}-n}{\sqrt{n+\sqrt{n}}+\sqrt{n}} = \dfrac{\sqrt{n}}{\sqrt{n+\sqrt{n}}+\sqrt{n}} 
		\]
		Durch Ausklammern von $\sqrt{n}$ im Nenner folgt:
		\[
			= \dfrac{\sqrt{n}}{\sqrt{n}\cdot \left(\sqrt{1+\dfrac{1}{\sqrt{n}}}+1\right)} = \dfrac{1}{\sqrt{1+\dfrac{1}{\sqrt{n}}}+1}
		\]
		Nun gilt allgemein, da es sich bei $n$ um eine positive Zahl handelt (inverse der Wurzel existiert und muss größer Null sein):
		\[
			1+\dfrac{1}{\sqrt{n}} > 1 \ \Rightarrow \ \sqrt{1+\dfrac{1}{\sqrt{n}}} > \sqrt{1} = 1 \ \Rightarrow \ \sqrt{1+\dfrac{1}{\sqrt{n}}} + 1 > 1+1 = 2
		\]
		Beide Terme sind wieder größer Null. Bildet man das Inverse dieser beiden Terme, muss sich also das Ungleichheitszeichen umkehren.
		\[
			\Rightarrow \ \dfrac{1}{\sqrt{1+\dfrac{1}{\sqrt{n}}}+1} = b_n < \dfrac{1}{2} \ \Rightarrow \ b_n -\dfrac{1}{2} < 0
		\]
		Durch Multiplikation mit $-1$ (bzw. auch durch Anwendung der Betragsfunktion) erhält man:
		\[
			\Rightarrow \ \dfrac{1}{2}-b_n > 0 	\ \Rightarrow \ b_n < \dfrac{1}{2}
		\] 
		Weiterhin gilt aber:
		\[
			\dfrac{1}{2}-b_n = \dfrac{1}{2} - \dfrac{1}{\sqrt{1+\dfrac{1}{\sqrt{n}}}+1} = \dfrac{\sqrt{1+\dfrac{1}{\sqrt{n}}}+1 - 2}{2+2\cdot\sqrt{1+\dfrac{1}{\sqrt{n}}}} = \dfrac{\sqrt{1+\dfrac{1}{\sqrt{n}}} - 1}{2+2\cdot\sqrt{1+\dfrac{1}{\sqrt{n}}}}
		\]
		Weil $n > 0$ gilt, muss folgen:
		\[
			2+2\cdot\sqrt{1+\dfrac{1}{\sqrt{n}}} > 2 > 1 \ \Rightarrow \ \dfrac{1}{2+2\cdot\sqrt{1+\dfrac{1}{\sqrt{n}}}} < 1
		\]
		Diese Ungleichung benutzt man für die eben aufgestellte Gleichung:
		\[
			\dfrac{1}{2}-b_n = \dfrac{\sqrt{1+\dfrac{1}{\sqrt{n}}} - 1}{2+2\cdot\sqrt{1+\dfrac{1}{\sqrt{n}}}} \ < \ \sqrt{1+\dfrac{1}{\sqrt{n}}} - 1 \ < \  1+\dfrac{1}{\sqrt{n}}-1=\dfrac{1}{\sqrt{n}}
		\]
		In der Wurzel der vorletzten Ungleichung, wird $1$ mit dem Inversen der Wurzel von $n$ addiert. Diese Addition muss eine Zahl ergeben, welche größer $1$ ist. Damit muss also das Quadrat dieser Wurzel auch größer als die Wurzel selbst sein.
		\[
			\Rightarrow \ \dfrac{1}{2}-b_n < \dfrac{1}{\sqrt{n}} \ \Rightarrow \ \dfrac{1}{2}-\dfrac{1}{\sqrt{n}} < b_n
		\]
		Allgemein also:
		\[
			\dfrac{1}{2}-\dfrac{1}{\sqrt{n}} < b_n < \dfrac{1}{2}
		\]
		Da $1/\sqrt{n} \longrightarrow 0$ gilt muss sowohl die linke als auch die rechte Folge gegen $1/2$ konvergieren. Nach dem Sandwich-Theorem muss also auch
		\[
			b_n \longrightarrow \dfrac{1}{2}, n\longrightarrow \infty
		\]
		gelten.
		Für die Folge $(c_n)$ folgt (wieder durch Ausklammern von $\sqrt{n}$):

		\[
			c_n = \sqrt{n+\frac{n}{a}}-\sqrt{n} = \sqrt{n}\cdot \left( \sqrt{1+\dfrac{1}{a}} -1 \right)
		\]

		Der rechte Faktor ist unabhängig von $n$. Damit ist er also eine Konstante. Diese Konstante soll mit $m$ substituiert werden.

		\[
			m := \sqrt{1+\dfrac{1}{a}} -1 > 0
		\]

		Das Inverse von $a$ muss, weil $a$ positiv ist, größer Null sein. Addiert man nun in der Wurzel eine Zahl größer Null zu $1$ hinzu, so muss die Wurzel ebenfalls größer $1$ sein. Da $m >0 $ aus dieser Betrachtung folgt, ist das Inverse von $m$ (hier $m^{-1}$) ebenfalls größer Null. Es folgt:
		\[
			c_n = \sqrt{n}\cdot m = \dfrac{\sqrt{n}}{m^{-1}} > 0
		\]
		Damit handelt es sich um das Reziproke der am Anfang behandelten Folge, welche gegen $0$ konvergiert. In der Vorlesung wurde dann gezeigt, dass das Reziproke bestimmt divergent gegen $\infty$ ist. Es gilt also:
		\[
			c_n \rightarrow \infty, n \rightarrow \infty
		\] 
		$\hfill\Box$
		
		\newpage

	% subsection aufgabe_1 (end)


	\subsection*{Aufgabe 2} % (fold)
	\label{sub:aufgabe_2}
	
		\subsubsection*{a)} % (fold)
		\label{ssub:a_}
		
			\begin{description}
				\item[(i):] Für alle $k\in \mathbb{N}$ gibt es ein $n_0 \in \mathbb{N}$, sodass für alle $n \geq n_0$ $\ $ $\ $ $|x_n-x| < \frac{1}{k}$ gilt.
				\item[(ii):] Für alle $q\in \mathbb{Q}\setminus\{0\}$ gibt es ein $n_0 \in \mathbb{N}$, sodass für alle $n \geq n_0$ $\ $ $\ $ $|x_n-x| < q^2$ gilt.
				\item[(iii):] Für alle $\varepsilon > 0$ gibt es ein $n_0 \in \mathbb{N}$, sodass für alle $n \geq n_0$ $\ $ $\ $ $|x_n-x| \leq \varepsilon$ gilt.
				\item[(iv):] Es gibt ein $n_0 \in \mathbb{N}$ für alle $\varepsilon > 0$, sodass für alle $n \geq n_0$ $\ $ $\ $ $|x_n-x| < \varepsilon$ gilt.
				\item[(v):] Für alle $n_0 \in \mathbb{N}$ gibt es ein $\varepsilon > 0$, sodass für alle $n \geq n_0$ $\ $ $\ $ $|x_n-x| < \varepsilon$ gilt.
			\end{description}

		% subsubsection a_ (end)

		\subsubsection*{b)} % (fold)
		\label{ssub:b_}
		
			Die Folge $(x_n)$ in $\mathbb{R}$ konvergiert gegen $x \in\mathbb{R}$, wenn zu jedem $\epsilon > 0$ ein $n_{\epsilon} \in \mathbb{N}$ existiert, sodass für alle $n \geq n_{\epsilon} \ \ $ $|x_n-x|<\epsilon$ gilt. Drückt man dies mithilfe von Quantoren aus, folgt:

			\[
				\forall \  \epsilon > 0 \ \ \ \exists \ n_{\epsilon} \in \mathbb{N} \ \ \ \forall \ n \geq n_{\epsilon} \ \ \ |x_n-x|<\epsilon
			\]

			\begin{description}
				\item[(i):] Sei die Aussage wie in a). Dann gilt nach dem Archimedischen Axiom, dass es für alle $\varepsilon > 0$ ein $k\in \mathbb{N}$ mit $\frac{1}{k}<\varepsilon$ gibt. Dann gilt also für ein jeweiliges $\varepsilon$ immer:
				\[
					|x_n-x| < \frac{1}{k} < \varepsilon
				\]
				Damit gibt es also für alle $\varepsilon > 0$ ein $n_0 \in \mathbb{N}$, sodass für alle $n \geq n_0$ gilt: $|x_n-x| < \varepsilon$. Dies ist äquivalent zur Grenzwertdefinition. Damit muss die Aussage (i) dazu äquivalent sein, dass es sich bei $x$ um einen Grenzwert der Folge $(x_n)$ handelt.
				
				\item[(ii):] Sei die Aussage wie in a). Dann gilt für ein $m \in \mathbb{Z}\setminus \{0\}$ und ein $n \in \mathbb{N}$:
				\[
					q = \dfrac{m}{n} \ \Rightarrow \ q^2 = \dfrac{m^2}{n^2} 
				\]
				Es gilt, dass $m^2 > 0$. Damit lassen sich sowohl $m^2$ als auch $n^2$ mit natürlichen Zahlen substituieren.
				Betrachtet man wieder das Archimedische Axiom, dann gilt für die gleichen Definitionen wie gerade eben, $\frac{1}{k}<\varepsilon$. Durch Multiplikation mit einer weiteren natürlichen Zahl $p$ folgt:
				\[
					\dfrac{p}{k} < p\cdot\varepsilon
				\]
				Sei nun ein $\varepsilon^\prime > p\cdot\varepsilon > 0$, dann gilt, wenn man $p:=m^2$ und $k:=n^2$ setzt:
				\[
					\dfrac{p}{k} = \dfrac{m^2}{n^2} = q^2 < \varepsilon^\prime \ \Rightarrow \ |x_n-x| < q^2 < \varepsilon^\prime
				\]
				Damit gibt es also für alle $\varepsilon^\prime > 0$ ein $n_0 \in \mathbb{N}$, sodass für alle $n \geq n_0$ gilt: $|x_n-x| < \varepsilon^\prime$. Dies ist äquivalent zur Grenzwertdefinition. Damit muss die Aussage (iii) dazu äquivalent sein, dass es sich bei $x$ um einen Grenzwert der Folge $(x_n)$ handelt.

				\item[(iii):] Sei die Aussage wie in a). Dann findet man zu jedem $\varepsilon^\prime > 0$ auch ein $\varepsilon > 0$ mit $\varepsilon < \varepsilon^\prime$. Dann folgt:
				\[
					|x_n-x| \leq \varepsilon < \varepsilon^\prime
				\]
				Damit gibt es also für alle $\varepsilon^\prime > 0$ ein $n_0 \in \mathbb{N}$, sodass für alle $n \geq n_0$ gilt: $|x_n-x| < \varepsilon^\prime$. Dies ist äquivalent zur Grenzwertdefinition. Damit muss die Aussage (iii) dazu äquivalent sein, dass es sich bei $x$ um einen Grenzwert der Folge $(x_n)$ handelt.

				\item[(iv):] Sei die Aussage wie in a). Dann ist diese Aussage nicht äquivalent zur Grenzwertdefinition. Nimmt man als Beispiel die Folge $1/n$. Dann muss auch für alle $\varepsilon > 0$ gelten:
				\[
					\left| \dfrac{1}{n_0} - 0 \right| = \dfrac{1}{n_0} < \varepsilon
				\]
				Wählt man allerdings $\varepsilon = 1/2n_0$. Dann gilt:
				\[
					\varepsilon < \dfrac{1}{n_0}
				\]
				Dies ruft also einen Widerspruch hervor. Damit würde die Folge $1/n$ nach dieser Aussage nicht den Grenzwert Null besitzen. Aber $1/n$ besitzt nach der eigentliche Definition diesen Grenzwert. Die Aussage kann also nicht äquivalent zur Grenzwertdefinition sein.

				\item[(v):] Sei die Aussage wie in a). Dann ist diese Aussage nicht äquivalent zur Grenzwertdefinition. Nimmt man als Beispiel die Folge $(-1)^n$, welche eigentlich keinen Grenzwert besitzt. Dann gilt für alle $n_0 \in \mathbb{N}$:
				\[
					|(-1)^{n_0} - 0| < 2
				\]
				Damit gibt es also zu jedem $n_0$ ein $\varepsilon > 0$ mit $\varepsilon = 2$, sodass für alle $n \geq n_0$ die Aussage gilt, sofern es sich bei dem Grenzwert um Null handelt. Damit besitzt die Beispielfolge nach dieser Aussage einen Grenzwert. Die Aussage kann also nicht äquivalent zur Grenzwertdefinition sein.
			\end{description}

		% subsubsection b_ (end)

		\newpage

	% subsection aufgabe_2 (end)


	\subsection*{Aufgabe 3} % (fold)
	\label{sub:aufgabe_3}

		\subsubsection*{a)} % (fold)
		\label{ssub:a_}
		
			\begin{description}
				\item[Behauptung:] \hfill \\
					$a_n = \frac{1}{n}\cdot \left( (n+1)^2 -n^2 \right) \ \longrightarrow \ 2$
				\item[Beweis:]
			\end{description}
			
			Sei die Folge $(a_n)$ wie in der Behauptung gegeben. Angenommen der Grenzwert dieser Folge sei $a := 2$. Durch äquivalente Umformungen von $(a_n)$ ergibt sich:

			\[
				a_n = \frac{1}{n}\cdot \left( (n+1)^2 -n^2 \right) = \frac{1}{n}\cdot \left( n^2 + 2n + 1 -n^2 \right) = \dfrac{1}{n}\cdot ( 2n + 1) = 2 + \dfrac{1}{n}
			\]

			Sei nun ein $\varepsilon > 0$. Dann gibt es ein $n_{\varepsilon} \in \mathbb{N}$, sodass für alle $n \geq n_{\varepsilon}$ gilt:

			\[
				|a_n - a| < \varepsilon
			\]

			Es gilt:

			\[
				|a_n-a| = \left|2+\dfrac{1}{n} - 2\right| = \left| \dfrac{1}{n} \right| = \dfrac{1}{n} < \varepsilon
			\]

			Nach dem Archimedischen Axiom gibt es für alle $\varepsilon > 0$ ein $k \in \mathbb{N}$, sodass $1/k < \varepsilon$ gilt. Wird also das $n_{\varepsilon}$ so gewählt, dass dies erfüllt ist, gilt:

			\[
				\dfrac{1}{n_{\varepsilon}} < \varepsilon 
			\]

			Für jedes $n \geq n_{\varepsilon}$ folgt:

			\[
				n \geq n_{\varepsilon} \ \Rightarrow \  \dfrac{1}{n} \leq \dfrac{1}{n_{\varepsilon}}  \ \Rightarrow \ \dfrac{1}{n} < \varepsilon 
			\]

			Also gilt, dann für alle $n \geq n_{\varepsilon}$:

			\[
				|a_n-a| = \dfrac{1}{n} < \varepsilon
			\]

			Damit ist gezeigt, dass $a_n \longrightarrow 2$ gilt. $\hfill\Box$

		% subsubsection a_ (end)

		\subsubsection*{b)} % (fold)
		\label{ssub:b_}
		
			\begin{description}
				\item[Behauptung:] \hfill \\
					$b_n = \frac{n}{2^n} \ \longrightarrow \ 0$
				\item[Beweis:]
			\end{description}
			
			Grundsätzlich wurde dieser Beweis bereits in der Vorlesung erbracht.\\
			Sei $0 < q < 1$. Dann gibt es nach dem Archimedischen Axiom für alle $\varepsilon > 0$ ein $n_{\varepsilon} \in \mathbb{N}$ mit $0 < q^{n_{\varepsilon}} < \varepsilon$. Für alle $n \geq n_{\varepsilon}$ folgt dann:
			\[
				0 < \ q^n = \left|q^n-0\right| \ < q^{n_{\varepsilon}} < \varepsilon
			\]
			Dabei handelt es sich um die Beschreibung eines Grenzwertes. Es gilt $q^n \longrightarrow 0, n\longrightarrow \infty$ für ein $0 < q < 1$.\\

			Für die Folge $(b_n)$ folgt dann:
			\[
				b_n = \frac{n}{2^n} = n\cdot \left(\dfrac{1}{2}\right)^n = n \cdot \left( \dfrac{1}{\sqrt{2}} \right)^{2n} = n \cdot \left( \dfrac{1}{\sqrt{2}} \right)^{n} \cdot \left( \dfrac{1}{\sqrt{2}} \right)^{n} > 0
			\]
			Es muss sich hierbei um eine positive Folge handeln, da es sich bei allen Größen um positive Größen handelt.
			Weiterhin gilt:
			\[
				\sqrt{2}>1>0  \ \Rightarrow \  0 < \dfrac{1}{\sqrt{2}} < 1 \ \Rightarrow \ \left( \dfrac{1}{\sqrt{2}} \right)^{n} \longrightarrow 0
			\]
			Für ein $a > 0$ folgt aufgrund der Bernoulli-Ungleichung:
			\[
				\sqrt{2} = 1+a \ \Rightarrow \ \left( \sqrt{2} \right)^n = (1+a)^n \geq 1 + an  \ \Rightarrow \  \left(\dfrac{1}{\sqrt{2}}\right)^n \leq \dfrac{1}{1+an} < \dfrac{1}{an} 
			\]

			Damit gilt für $(b_n)$:
			\[
				b_n = n \cdot \left( \dfrac{1}{\sqrt{2}} \right)^{n} \cdot \left( \dfrac{1}{\sqrt{2}} \right)^{n} < n\cdot \dfrac{1}{an} \cdot \left( \dfrac{1}{\sqrt{2}} \right)^{n} = \dfrac{1}{a} \cdot \left(\dfrac{1}{\sqrt{2}}\right)^n
			\]
			\[
				\Rightarrow \ \dfrac{1}{a} \cdot \left(\dfrac{1}{\sqrt{2}}\right)^n \ \longrightarrow \ \dfrac{1}{a} \cdot 0 = 0 
			\]
			\[
				\Rightarrow \ 0 < b_n < \dfrac{1}{a} \cdot \left(\dfrac{1}{\sqrt{2}}\right)^n
			\]

			Nach dem Sandwich-Theorem folgt also wieder:
			\[
				b_n\longrightarrow 0,n\longrightarrow \infty
			\]
			$\hfill\Box$

		% subsubsection b_ (end)

		\subsubsection*{c)} % (fold)
		\label{ssub:c_}
		
			\begin{description}
				\item[Behauptung:] \hfill \\
					$c_n = \frac{1}{n^4}\cdot \sum_{k=1}^n k^3 \longrightarrow \frac{1}{4}$
				\item[Beweis:]
			\end{description}
			
			Beweis des Hinweises durch Induktion:

			\underline{Induktionsanfang für $n=1$:}\\
			\[
				\sum_{k=1}^n = \sum_{k=1}^1 k^3 = 1^3 = 1 = 1^2\dfrac{4}{4}= \dfrac{1}{4}1^22^2 = \dfrac{1}{4}1^2(1+1)^2 = \dfrac{1}{4}n^2(n+1)^2
			\]
			Damit ist die Gleichung für $n=1$ gezeigt.\\

			\underline{Induktionsvoraussetzung:} $\sum_{k=1}^n k^3 = \dfrac{1}{4}n^2(n+1)^2$ \\

			\underline{Induktionsbehauptung:} $\sum_{k=1}^{n+1} k^3 = \dfrac{1}{4}(n+1)^2(n+2)^2$\\

			\underline{Induktionsschluss:}
			\[
				\sum_{k=1}^{n+1} k^3 = (n+1)^3 + \sum_{k=1}^n k^3
			\]
			Wegen der Induktionsbehauptung gilt:
			\[
				= (n+1)^3 + \dfrac{1}{4}n^2(n+1)^2
			\]
			Durch Ausklammern von $\frac{1}{4}(n+1)^2$ folgt:
			\[
				= \dfrac{1}{4}(n+1)^2 \cdot (4\cdot(n+1) + n^2) = \dfrac{1}{4}(n+1)^2 \cdot (n^2 + 4n + 4)
			\]
			Für den letzten Faktor lässt sich nun die bereits gezeigte erste binomische Formel verwenden:
			\[
				= \dfrac{1}{4}(n+1)^2(n+2)^2
			\]
			Damit wurde die Behauptung gezeigt. Die Formel gilt also allgemein für alle $n \in \mathbb{N}$. \\

			Für die Folge $(c_n)$ folgt dann, da es sich bei allen Größen wieder um positive Zahlen handelt:
			\[
				c_n = \frac{1}{n^4}\cdot \sum_{k=1}^n k^3 = \dfrac{n^2(n+1)^2}{4n^4} = \dfrac{(n+1)^2}{4n^2} = \dfrac{n^2 + 2n + 1}{4n^2} = \dfrac{1}{4} + \dfrac{1}{2n} + \dfrac{1}{4n^2} > 0
			\]
			Die Folge $1/2n$ und die Folge $1/4n^2$ konvergieren beide gegen den Wert Null (dies wurde bereits in der Vorlesung gezeigt). Damit muss auch ihre Summe gegen Null konvergieren. Dann bedeutet dies, dass es für alle $\varepsilon > 0$ ein $n_{\varepsilon} \in \mathbb{N}$ gibt, sodass für alle $n \geq n_{\varepsilon}$ gilt:
			\[
				\dfrac{1}{2n} + \dfrac{1}{4n^2} < \varepsilon
			\]
			Es gilt dann also:
			\[
				\dfrac{1}{2n} + \dfrac{1}{4n^2} = c_n - \dfrac{1}{4} = \left| c_n -\dfrac{1}{4} \right| < \varepsilon
			\]
			Dies ist genau die Definition eines Grenzwertes für $(c_n)$. Es folgt also:
			\[
				c_n \longrightarrow \dfrac{1}{4}, n \longrightarrow \infty
			\]
			$\hfill\Box$

		% subsubsection c_ (end)
	
		\newpage

	% subsection aufgabe_3 (end)


	\subsection*{Aufgabe 4} % (fold)
	\label{sub:aufgabe_4}

		Auch hier soll wieder der bereits bewiesene Satz der Vorlesung benutzt werden. Für alle $k \in \mathbb{R}$ und für $n \in \mathbb{N}$ gilt $k/n \rightarrow 0, n \rightarrow \infty$.
		Durch Umformung der einzelnen Folgen in eine Zusammensetzung aus $1/n$ kann also der Grenzwert ermittelt werden. Hierbei werden die bekannten Rechenregeln benutzt, dass der Grenzwert für die Addition/Subtraktion/Multiplikation von zwei Folgen die Addition/Subtraktion/Multiplikation der beiden Grenzwerte der Folgen ist.

		\subsubsection*{a)} % (fold)
		\label{ssub:a_}
			
			Zuerst soll das Inverse gebildet werden. Nun ist es möglich den Nenner auszumultiplizieren. Durch Ausklammern von $1/n$ erhält man dann eine gewünschte Zusammensetzung:
			\[
				\dfrac{(n+1)!}{(n+2)!-n!} = \dfrac{1}{\dfrac{(n+2)!-n!}{(n+1)!}} =  \dfrac{1}{n+2 - \dfrac{1}{n+1}} = \dfrac{1}{n}\cdot \dfrac{1}{1+\dfrac{2}{n} + \dfrac{1}{n}\cdot\dfrac{1}{n+1}}
			\]
			\[
				= \dfrac{1}{n}\cdot \dfrac{1}{1+\dfrac{2}{n} + \dfrac{1}{n}\cdot\dfrac{1}{n}\cdot \dfrac{1}{1+\dfrac{1}{n}}} \ \longrightarrow \ 0\cdot \dfrac{1}{1+0+0\cdot 0\cdot \dfrac{1}{1+0}} = 0\cdot 1 = 0
			\]
			Diese Folge besitzt also einen Grenzwert und ist damit konvergent.

		% subsubsection a_ (end)

		\subsubsection*{b)} % (fold)
		\label{ssub:b_}
		
			\[
				(-1)^n\cdot\dfrac{n^2}{2n^2+5} = (-1)^n\cdot \dfrac{1}{2+\dfrac{5}{n^2}}
			\]
			$(-1)^n$ ist nicht konvergent oder bestimmt divergent. Diese Folge wechselt ständig ihren Wert zwischen $1$ und $-1$. Damit kann die gesamte nur einen Grenzwert besitzen, wenn diese Folge durch Multiplikation mit $0$ neutralisiert wird. Die gesamte Folge müsste also gegen den Wert $0$ konvergieren, um überhaupt einen Grenzwert zu besitzen. Damit kann die genannte Folge nicht bestimmt divergent sein. Die gesamte Folge kann also nur konvergent sein, wenn der rechte Faktor gegen Null konvergiert. 
			\[
				\dfrac{1}{2+\dfrac{5}{n^2}} \ \longrightarrow \ \dfrac{1}{2+0} = \dfrac{1}{2} \neq 0
			\]
			Da dieser allerdings nicht gegen Null konvergiert, kann es für die gesamte Folge auch keinen Grenzwert geben. Die Folge ist also nicht konvergent oder bestimmt divergent.
			

		% subsubsection b_ (end)

		\subsubsection*{c)} % (fold)
		\label{ssub:c_}
			
			Aus dem Zähler soll $n^2$ und aus dem Nenner soll $n^3$ ausgeklammert werden. Damit folgt für die Folge:
			\[
				\dfrac{3n^2+n}{n^3+n-1} = \dfrac{n^2\cdot \left(3+\dfrac{1}{n}\right)}{n^3\cdot \left( 1+\dfrac{1}{n^2} -\dfrac{1}{n^3} \right)} = \dfrac{1}{n}\cdot\dfrac{3+\dfrac{1}{n}}{1+\dfrac{1}{n^2} -\dfrac{1}{n^3}} \ \longrightarrow \ 0\cdot\dfrac{3+0}{1+0-0} = 0
			\]
			Diese Folge ist also gegen Null konvergent.

		% subsubsection c_ (end)
		
		\newpage

	% subsection aufgabe_4 (end)

	% section 1 (end)

\end{document}

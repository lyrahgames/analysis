\documentclass[11pt, a4paper]{article}

\usepackage[utf8]{inputenc}

\usepackage{amsmath}
\usepackage{amssymb}
\usepackage{float}

\usepackage{geometry}
\geometry{a4paper, lmargin=30mm, rmargin=20mm, tmargin=30mm, bmargin=20mm}

\setlength{\parindent}{0mm}

\begin{document}
	\begin{center} \section*{Analysis I - Übungsserie 4} \end{center}

	Übungsgruppe: Jonas Franke \\


	Nina Held: 144753 \\
	Clemens Anschütz: 146390 \\
	Markus Pawellek: 144645 \\


	\subsection*{Aufgabe 1}

		\begin{description}
			\item[Voraussetzung:] \hfill \\
				Für $A,B \subset \mathbb{R}$ seien \\
				$A+B \ := \ \{ x\in \mathbb{R} \ | \ \text{es existieren } a \in A \text{ und } b \in B \text{ mit } x = a+b \}$\\
				$A\cdot B \ := \ \{ x\in \mathbb{R} \ | \ \text{es existieren } a \in A \text{ und } b \in B \text{ mit } x = a\cdot b \}$. \hfill 
		\end{description}

	\subsubsection*{(a)}

		\begin{description}
			\item[Voraussetzung:] \hfill \\
				$A,B \subset \mathbb{R}$ sind nach unten beschränkt. \hfill 
			\item[Behauptung:] \hfill \\
				$inf(A+B) = inf(A) + inf(B)$ gilt. \hfill
			\item[Beweis:] 
		\end{description}

		$A$ und $B$ sollen die in der Voraussetzung beschriebenen Mengen sein. Dann sind es nach unten beschränkte Mengen. Da $\mathbb{R}$ ein angeordneter Körper ist, müssen sie deshalb beide ein Infimum besitzen.
		Damit gilt für alle $a \in A$ und alle $b \in B$:

		\[ a \geq inf(A) \]
		\[ b \geq inf(B) \]

		Aufgrund der Ordnungsvollständigkeit gilt dann die Ungleichung auch bei Addition der beiden Terme. Für alle $a \in A$ und $b \in B$ gilt:

		\[ a+b \geq inf(A)+ inf(B) \]

		Aufgrund der Definition von $A+B$ gilt dann für alle $x \in A+B$:

		\[ x \geq inf(A) + inf(B) \]

		Nach Definition muss dies eine untere Schranke von $A+B$ sein. Auch hier folgt wegen der Ordnungsvollständigkeit, dass $A+B$ ein Infimum besitzt.
		Jede andere untere Schranke von $A$ oder $B$ ist kleiner als deren Infimum. Es folgt für eine beliebige untere Schranke $s_a \in \mathbb{R}$ von $A$ und $s_b \in \mathbb{R}$ von $B$:

		\[ inf(A) \geq s_a \]
		\[ inf(B) \geq s_b \]

		Damit gilt auch hier für alle $x \in A+B$:

		\[ x \geq inf(A) + inf(B) \geq s_a + s_b \]

		\newpage

		Die Addition der unteren Schranken von $A$ und $B$ ergibt also die unteren Schranken von $A+B$. Damit ist die untere Schranke $inf(A) + inf(B)$ die größte untere Schranke von $A+B$. Nach der Definition gilt also für das Infimum von $A+B$:

		\[ inf(A+B) = inf(A) + inf(B) \]
		$\hfill\Box$


	\subsubsection*{(b)}

		\begin{description}
			\item[Voraussetzung:] \hfill \\
				$A,B \subset \mathbb{R}$ sind beschränkt. $A,B \neq \varnothing$ \hfill 
			\item[Behauptung:] \hfill \\
				Es gibt $A,B$ mit $inf(A\cdot B) \neq inf(A) \cdot inf(B)$ \hfill
			\item[Beweis:] 
		\end{description}

		Seien $A,B$ die in den Voraussetzungen beschriebenen Mengen mit 

		\[ A := \{ 1,2 \} \]
		\[ B := \{ -1,1 \} \]

		Beide Mengen sind damit beschränkt und eine Teilmenge von $\mathbb{R}$ (damit existieren auch ihre Infima). Es gilt dann:

		\[ inf(A) = 1 \]
		\[ inf(B) = -1 \]

		$A\cdot B$ ergibt sich mit dem Infimum zu:

		\[ A\cdot B = \{-2,-1,1,2 \} \]
		\[ inf(A\cdot B) = -2 \]

		Es folgt:

		\[ inf(A) \cdot inf(B) = 1\cdot (-1) = -1 \neq inf(A\cdot B) = -2 \]

		Damit existieren mindestens diese Mengen $A,B$, für welche die Behauptung erfüllt ist. $\hfill\Box$


	\newpage

	\subsection*{Aufgabe 2}

		\begin{description}
			\item[Voraussetzung:]\hfill \\
				Es sei die Abbildung $f:\mathbb{N}\longrightarrow \mathbb{R}$ mit $f(m+n)= f(m) + f(n) + a$ für alle $m,n \in \mathbb{N}$ und ein $a \in \mathbb{R}$. Es gilt $f(2) = 10$ und $f(20) = 118$. \hfill
			\item[Aufgabe:]\hfill \\
				Ist $f$ eindeutig? Wenn ja, was ist $a$ und $f$?
			\item[Lösung:]
		\end{description}

		Seien alle Variablen und Abbildungen wie in den Voraussetzungen definiert. Dann gilt für alle $n \in \mathbb{N}$:

		\[ f(n) = f((n-1)+1) \]

		Aufgrund der Definition folgt nun:

		\[ = f(n-1) + f(1) + a \]
		\[ = f((n-2)+1) + f(1) + a = f(n-2) + f(1) + a + f(1) + a = f(n-2) + 2\cdot f(1) + 2\cdot a \]

		Folgende Vermutung ergibt sich also:

		\[ f(n) = f(1) + (n-1)\cdot f(1) + (n-1)\cdot a = n\cdot f(1) + (n-1)\cdot a \] \\



		Beweis dieser Schlussfolgerung durch Induktion:\\



		\underline{Induktionsanfang für $n=1$}:

		\[ f(1) = 1\cdot f(1) + (1-1)\cdot a = f(1) + 0\cdot a = f(1) \]

		Damit ist Behauptung für $n=1$ erfüllt.\\



		\underline{Induktionsvoraussetzung:} $f(n) = n\cdot f(1) + (n-1)\cdot a$\\


		\underline{Induktionsbehauptung:} $f(n+1) = (n+1)\cdot f(1) + n\cdot a$ \\


		\underline{Induktionsschluss:}\\
		Durch Anwendung der Funktionsbedingung ergibt sich:

		\[ f(n+1) = f(n) + f(1) + a \]

		Durch Einsetzen der Induktionsvoraussetzung folgt:

		\[ = n\cdot f(1) + (n-1)\cdot a + f(1) + a = (n+1)\cdot f(1) + n\cdot a \]

		Damit wäre die Induktionsbehauptung bewiesen.\\


		Damit gilt also für alle $m,n \in \mathbb{N}$ und ein $a \in \mathbb{R}$:

		\[ f(m+n)= f(m) + f(n) + a \ \Rightarrow \ f(n) = n\cdot f(1) + (n-1)\cdot a \] 

		Ist nun aber $f(n) = n\cdot f(1) + (n-1)\cdot a$ für alle $n \in \mathbb{N}$ und ein $a \in \mathbb{R}$ als Bedingung für die Funktion in den Voraussetzungen gegeben, folgt für alle $m,n \in \mathbb{N}$:

		\[ f(m+n) = (m+n)\cdot f(1) + (m+n-1)\cdot a \]
		\[ = m\cdot f(1) + n\cdot f(1) + m\cdot a + (n-1)\cdot a \]
		\[ = m\cdot f(1) + n\cdot f(1) + (m-1)\cdot a + (n-1)\cdot a + a \]
		\[ = (m\cdot f(1) + (m-1)\cdot a) + (n\cdot f(1) + (n-1)\cdot a) + a \]
		\[ = f(m) + f(n) + a \]

		Damit folgt auch für alle $m,n \in \mathbb{N}$ und ein $a \in \mathbb{R}$:

		\[ f(n) = n\cdot f(1) + (n-1)\cdot a \ \Leftrightarrow \ f(m+n) = f(m) + f(n) + a \]

		Es ist also eine Äquivalenz zwischen den beiden Funktionsbedingungen vorhanden. Beweist man also die Eindeutigkeit für eine Bedingung, ist sie auch für die andere Bedingung bewiesen. Damit muss $f$ eindeutig sein, wenn $f(1)$ und $a$ eindeutig durch äquivalente Umformungen bestimmt werden können. Aus den weiteren Bedingungen folgt:

		\[ f(2) = 10 = 2\cdot f(1) + a \]
		\[ \Rightarrow \ a = 10 - 2\cdot f(1) \]

		Weiterhin gilt:

		\[ f(20) = 118 = 20\cdot f(1) + 19\cdot a \]
		\[ = 20\cdot f(1) + 19\cdot (10 - 2\cdot f(1)) \]
		\[ = (-18)\cdot f(1) + 190 \]


		\[ \Rightarrow \ f(1) = \dfrac{118-190}{-18} = 4 \]
		\[ \Rightarrow \ a = 10 - 2\cdot 4 = 2 \]


		\[ \Rightarrow \ f(n) = 4n + 2\cdot (n-1) = 6n - 2 \]

		Damit konnten durch äquivalente Umformungen $f(1)$ und $a$ eindeutig bestimmt werden. Es muss also auch $f$ durch die Bedingungen eindeutig sein.


	\newpage

	\subsection*{Aufgabe 3}


		\begin{description}
			\item[Behauptung:]\hfill \\
				Für alle $n \in \mathbb{N}$ und alle $x_i \in \mathbb{R}$ mit $x_i > 0$ für $i \in \mathbb{N}$ und $1\leq i \leq n$ gilt: \\
				$\prod_{i=1}^n x_i = 1 \ \Longrightarrow \ \sum_{i=1}^n x_i \geq n$ \hfill
			\item[Beweis:]
		\end{description}

		Beweis soll durch Induktion geführt werden. Dabei seien alle Variablen wie in der Behauptung definiert.\\

		\underline{Induktionsanfang für $n=1$:}

			\[ x_1 = 1 \Rightarrow x_1 \geq 1 \]

			Damit muss also die Gleichung für $n=1$ erfüllt sein.\\

		\underline{Induktionsvoraussetzung:} $\prod_{i=1}^n x_i = 1 \ \Longrightarrow \ \sum_{i=1}^n x_i \geq n$\\


		\underline{Induktionsbehauptung:} $\prod_{i=1}^{n+1} x_i = 1 \ \Longrightarrow \ \sum_{i=1}^{n+1} x_i \geq n+1$ \\


		\underline{Induktionsschluss:}

		\[ 1 = \prod_{i=1}^{n+1} x_i = x_nx_{n+1}\cdot \prod_{i=1}^{n-1} x_i \]

		Definiert man jetzt für alle $i \in \mathbb{N}$ mit $1 \leq i \leq n-1$

		\[ y_i := x_i \]

		und für $n$

		\[ y_n := x_nx_{n+1} \]

		folgt:

		\[ x_nx_{n+1}\cdot \prod_{i=1}^{n-1} x_i = y_n \cdot \prod_{i=1}^{n-1} y_i = \prod_{i=1}^{n} y_i = 1 \]

		Dieses Produkt muss immer noch $1$ sein, da nur eine Größe durch eine andere ersetzt wurde. Für diese Gleichung folgt also aus der Induktionsvoraussetzung:

		\[ \Rightarrow \ \sum_{i=1}^n y_i \geq n \]
		\[ \Rightarrow \ x_nx_{n+1} + \sum_{i=1}^{n-1} x_i \geq n \]

		Durch Addition mit $1$ folgt:

		\[ \Rightarrow \ x_nx_{n+1} + 1 + \sum_{i=1}^{n-1} x_i \geq n+1 \]

		Es kann hier nun ohne Einschränkung angenommen werden, dass $x_{n+1} \geq 1$ und $x_n \leq 1$ ist. Dies folgt aus der Betrachtung, dass die Multiplikation aller $x_i$ gleich $1$ ergeben muss.
		Sind also Zahlen in diesem Produkt, welche größer $1$ sind, muss es auch mindestens eine Zahl geben, welche kleiner $1$ ist. Denn sonst würde es nicht möglich sein durch Multiplikation von Zahlen, welche alle größer $1$ sind, $1$ als Ergebnis zu erhalten. Der Spezialfall würde sich ergeben, wenn jeder Faktor in diesem Produkt $1$ wäre. Man findet also in diesem Produkt immer mindestens eine Zahl, welche größer gleich $1$ ist und eine andere, welche kleiner gleich $1$ ist. Da es sich bei den reellen Zahlen um einen angeordneten Körper handelt, können nun alle Faktoren beliebig vertauscht werden, ohne den Wahrheitswert der Gleichung zu ändern. Damit können auch $x_n$ und $x_{n+1}$ einen beliebigen Faktor dieses Produktes darstellen. \\


		Nun gilt für ein $\epsilon \in \mathbb{R}$ mit $\epsilon \geq 0$:

		\[ x_{n+1} \geq 1 \ \Rightarrow \ x_{n+1} = 1 + \epsilon \]

		Für ein $\epsilon' \in \mathbb{R}$ mit $0 \leq \epsilon' < 1$:

		\[ x_n \leq 1 \ \Rightarrow \ x_n = 1-\epsilon' \]

		Damit folgt 

		\[ x_nx_{n+1} + 1 = (1-\epsilon')\cdot(1+\epsilon) +1 = 1 + \epsilon -\epsilon' - \epsilon\epsilon' + 1 = 2 + \epsilon -\epsilon' - \epsilon\epsilon' \]

		Dabei ist zu beachten, da $\epsilon,\epsilon' \geq 0$, dass $\epsilon\epsilon' \geq 0$ gilt. Weiterhin gilt:

		\[ x_n + x_{n+1} = (1-\epsilon') + (1+ \epsilon) = 2 + \epsilon -\epsilon' \]


		\[ \Rightarrow \ 2 + \epsilon -\epsilon' \geq 2 + \epsilon -\epsilon' - \epsilon\epsilon' \]
		\[ \Rightarrow \ x_n + x_{n+1} \geq x_nx_{n+1} + 1 \]

		Aus den bereits umgestellten Ungleichungen folgt:

		\[ \Rightarrow \ x_n + x_{n+1} + \sum_{i=1}^{n-1} x_i \geq x_nx_{n+1} + 1 + \sum_{i=1}^{n-1} x_i \geq n+1 \]

		Durch Auslassen der mittleren Ungleichung folgt:

		\[ \Rightarrow \ \sum_{i=1}^{n+1} x_i \geq n+1 \]

		Damit folgt:

		\[ \prod_{i=1}^{n+1} x_i = 1 \ \Longrightarrow \ \sum_{i=1}^{n+1} x_i \geq n+1 \]

		Es wurde die Induktionsbehauptung gezeigt. Also muss auch die Behauptung gezeigt sein. $\hfill\Box$



	\newpage

	\subsection*{Aufgabe 4}

		\begin{description}
			\item[Behauptung:]\hfill \\
				Für alle $n \in \mathbb{N}$ und alle $a_i \in \mathbb{R}$ mit $a_i > 0$ für $i \in \mathbb{N}$ und $1\leq i \leq n$ gilt: \\
				\[ \dfrac{n}{\sum_{i=1}^n \dfrac{1}{a_i}} \ \leq \ \sqrt[n]{\prod_{i=1}^n a_i} \ \leq \ \dfrac{\sum_{i=1}^n a_i}{n} \] \hfill 
			\item[Beweis:]
		\end{description} 

		Seien die Variablen wie in den Voraussetzungen gegeben. Dann soll $p \in \mathbb{R}$ definiert sein als:

		\[ p := \sqrt[n]{a_1\cdot ... \cdot a_n} \geq 0 \]

		$p$ muss größer Null sein, da alle $a_i$ größer Null sind und durch Multiplikation zweier positiver Zahlen wieder eine positive Zahl herauskommt. 
		Seien $u_i$ für $i \in \mathbb{N}$ und $1\leq i \leq n$ definiert als:

		\[ u_i := \dfrac{p}{a_i} \]

		Damit gilt: 

		\[ \Rightarrow \ \prod_{i=1}^n u_i = \prod_{i=1}^n \dfrac{p}{a_i} = p^n \cdot \prod_{i=1}^n \dfrac{1}{a_i} = \dfrac{a_1\cdot ... \cdot a_n}{a_1\cdot ... \cdot a_n} = 1 \]

		Es muss also das Produkt aller $u_i$ gleich $1$ sein. Aus Aufgabe 3 folgt dann durch den bewiesenen Satz, dass Folgendes gilt:

		\[ \Rightarrow \ \sum_{i=1}^n u_i \geq n \]

		Durch Einsetzen ergibt sich:

		\[ \Rightarrow \ \sum_{i=1}^n \dfrac{p}{a_i} = p \cdot \sum_{i=1}^n \dfrac{1}{a_i} \geq n \]

		Jedes $a_i$ kommt als Inverses vor. Da jedoch alle $a_i > 0$ sind, müssen auch ihre Inversen größer Null sein. Damit gilt diese Ungleichung auch nach der Multiplikation mit der Summe der Inversen:

		\[ \Rightarrow \ p \geq \dfrac{n}{\sum_{i=1}^n \dfrac{1}{a_i}} \]

		Durch Einsetzen von $p$ folgt:

		\[ \Rightarrow \ \sqrt[n]{a_1\cdot ... \cdot a_n} = \sqrt[n]{\prod_{i=1}^n a_i} \geq \dfrac{n}{\sum_{i=1}^n \dfrac{1}{a_i}} \]

		Damit wäre die linke Seite der Ungleichung bewiesen. Für die rechte Seite der Ungleichung sollen $v_i$ für $i \in \mathbb{N}$ und $1\leq i \leq n$ definiert werden mit:

		\[ v_i := \dfrac{a_i}{p} \]

		Es gilt dann (ähnliche Betrachtungsweise wie oben):

		\[ \Rightarrow \ \prod_{i=1}^n v_i = \prod_{i=1}^n \dfrac{a_i}{p} = \dfrac{1}{p^n} \cdot \prod_{i=1}^n a_i = \dfrac{a_1\cdot ... \cdot a_n}{a_1\cdot ... \cdot a_n} = 1 \]

		Es ist also auch das Produkt der $v_i$ gleich $1$. Aus Aufgabe 3 folgt wieder durch den bewiesenen Satz:

		\[ \Rightarrow \ \sum_{i=1}^n v_i \geq n \]
		\[ \Rightarrow \ \sum_{i=1}^n \dfrac{a_i}{p} = \dfrac{1}{p} \cdot \sum_{i=1}^n a_i \geq n \]

		Es wurde bereits oben gezeigt, dass $p>0$. Außerdem gilt auch $n>0$, da es bereits so definiert worden ist. Es folgt also, dass auch $1/n > 0$ sein muss. Die Ungleichung gilt also auch nach der Multiplikation mit $p$ und $1/n$.

		\[ \Rightarrow \ \sum_{i=1}^n a_i \geq n\cdot p \]
		\[ \Rightarrow \ \dfrac{1}{n} \cdot \sum_{i=1}^n a_i \geq p \]

		Durch Einsetzen von $p$ folgt:

		\[ \Rightarrow \ \dfrac{1}{n} \cdot \sum_{i=1}^n a_i \geq \sqrt[n]{\prod_{i=1}^n a_i} = \sqrt[n]{a_1\cdot ... \cdot a_n} \]

		Damit wurde die rechte Seite der Ungleichung gezeigt und die Behauptung bewiesen. $\hfill\Box$


\end{document}
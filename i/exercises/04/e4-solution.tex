\documentclass[11pt, a4paper]{article}

\usepackage[utf8]{inputenc}

\usepackage{amsmath}
\usepackage{amssymb}
\usepackage{float}

\usepackage{geometry}
\geometry{a4paper, lmargin=30mm, rmargin=20mm, tmargin=30mm, bmargin=20mm}

\setlength{\parindent}{0mm}

\begin{document}

	\begin{center} \section*{Analysis I - Übungsserie 4} \end{center}

	Übungsgruppe: Jonas Franke \\


	Nina Held: 144753 \\
	Clemens Anschütz: 146390 \\
	Markus Pawellek: 144645 \\


	\subsection*{Aufgabe 1}

	Sei $(K,+,\cdot)$ ein angeordneter Körper. Dann gilt für alle $a,b \in K$ folgendes:

	\[ (a+b)^2 = (a+b)\cdot (a+b) = a\cdot(a+b) + b\cdot(a+b) = a\cdot a + a\cdot b + b\cdot a + b\cdot b \]
	\[ = a^2 + ab\cdot(1+1) + b^2 = a^2 + 2ab + b^2 \]

	Außerdem gilt nach einer analogen Herleitung:

	\[ (a-b)^2 = (a-b)\cdot(a-b) = a^2 -ab - ab + b^2 = a^2 -2ab+b^2 \]

	Damit gilt allgemein für alle $a,b \in K$:

	\[ (a+b)^2 = a^2 + 2ab + b^2 \]
	\[ (a-b)^2 = a^2 - 2ab + b^2 \]

	Des Weiteren folgt aus der bereits bewiesenen Proposition $x^2 \geq 0$ für alle $x \in K$:

	\[ (a+b)^2 \geq 0 \]
	\[ (a-b)^2 \geq 0 \]


	\subsubsection*{ a)}

		\begin{description}
			\item[Voraussetzung:] \hfill \\
				Sei $(K,+,\cdot)$ ein angeordneter Körper. \hfill 
			\item[Behauptung:] \hfill \\
				$|ab| \leq \dfrac{1}{2\lambda}a^2 + \dfrac{\lambda}{2}b^2$ für alle $a$,$b$,$\lambda$ $\in K$ mit $\lambda > 0$ \hfill 
			\item[Beweis:] 
		\end{description}

		Seien $a,b,\lambda$ wie in der Behauptung definiert. Dann gilt allgemein:

		\[ (a-\lambda b)^2 \geq 0 \]
		\[ a^2 - 2a\lambda b + \lambda^2b^2 \geq 0 \]

		Durch Addition von $2a\lambda b$ ergibt sich:

		\[ a^2 + \lambda^2b^2 \geq 2a\lambda b \]

		Durch Multiplikation mit $2^{-1}$ kann sich die Ungleichung nicht ändern, da $(2 > 0) \Rightarrow (2^{-1} > 0)$.
		Auch durch die Multiplikation mit $\lambda^{-1}$ kann keine Änderung auftreten, da $(\lambda > 0) \Rightarrow (\lambda^{-1} > 0)$.

		\[2^{-1}\cdot \lambda^{-1}\cdot a^2 + 2^{-1}\cdot \lambda^{-1} \cdot \lambda \cdot \lambda \cdot b^2 \geq 2^{-1}\cdot 2 \cdot \lambda^{-1} \cdot \lambda \cdot a\cdot b \]

		\[ \dfrac{1}{2\lambda}a^2 + \dfrac{\lambda}{2}b^2 \geq ab \]

		Weiterhin gilt:

		\[ (a+\lambda b)^2 \geq 0 \]
		\[ a^2 + 2a\lambda b + \lambda^2b^2 \geq 0 \]

		Durch Addition von $-2a\lambda b$ ergibt sich:

		\[ a^2 + \lambda^2b^2 \geq -2a\lambda b \]

		Durch Multiplikation mit $2^{-1}$ kann sich die Ungleichung nicht ändern, da $(2 > 0) \Rightarrow (2^{-1} > 0)$.
		Auch durch die Multiplikation mit $\lambda^{-1}$ kann keine Änderung auftreten, da $(\lambda > 0) \Rightarrow (\lambda^{-1} > 0)$.

		\[2^{-1}\cdot \lambda^{-1}\cdot a^2 + 2^{-1} \cdot \lambda^{-1} \cdot \lambda \cdot \lambda \cdot b^2 \geq (-1) \cdot 2^{-1}\cdot 2 \cdot \lambda^{-1} \cdot \lambda \cdot a\cdot b \]

		\[ \dfrac{1}{2\lambda}a^2 + \dfrac{\lambda}{2}b^2 \geq -ab \]

		Bei allen Umformungen handelte es sich um äquivalente Umformungen.\\

		Der Betrag von $ab$ kann entweder den Wert $-ab$ oder den Wert $ab$ annehmen. Für beide Fälle ist diese Ungleichung gezeigt worden. Damit gilt also allgemein:

		\[|ab| \leq \dfrac{1}{2\lambda}a^2 + \dfrac{\lambda}{2}b^2 \]
		$\hfill\Box$


	\subsubsection*{b)}

		\begin{description}
			\item[Voraussetzung:] \hfill \\
				Sei $(K,+,\cdot)$ ein angeordneter Körper. \hfill 
			\item[Behauptung:] \hfill \\
				$(a+b)^2 \geq 4ab$ für alle $a,b\in K$ \hfill 
			\item[Beweis:] 
		\end{description}

		Seien $a,b \in K$. Dann gilt:

		\[ (a-b)^2 \geq 0 \]

		Addiert man zu beiden Seiten $(a+b)^2$ folgt:

		\[ (a+b)^2 + (a-b)^2 \geq (a+b)^2 \]
		\[ (a+b)^2 + a^2 -2ab + b^2 \geq a^2 + 2ab + b^2 \]

		Durch Addition von $2ab$ ergibt sich:

		\[ (a+b)^2 + a^2 + b^2 \geq a^2 + 4ab + b^2 \]

		Nun addiert man $-a^2$ und $-b^2$:

		\[ (a+b)^2 \geq 4ab \]

		Damit wurde die Behauptung gezeigt, da es sich bei allen Umformungen um äquivalente Umformungen handelte. $\hfill\Box$ 

	\newpage

	\subsection*{Aufgabe 2}

		\begin{description}
			\item[Voraussetzung:] \hfill \\
				Sei $(K,+,\cdot)$ ein angeordneter Körper. \hfill 
			\item[Behauptung:] \hfill \\
				$\dfrac{r}{1+r} < \dfrac{s}{1+s}$ für alle $r,s\in K$ mit $0\leq r<s$ \hfill 
			\item[Beweis:] 
		\end{description}

		Seien $r,s$ wie in der Behauptung definiert. Dann gilt:

		\[ r < s \]

		Durch die Addition von $sr$ ergibt sich:

		\[ r+sr < s + sr \]
		\[ r\cdot (1+s) < s \cdot (1+r) \]

		Da $r,s \geq 0$ muss $(r+1),(s+1) > 0$ gelten. Damit müssen auch die Inversen $(r+1)^{-1}$ und $(s+1)^{-1}$ größer Null sein.
		Also ändert sich diese Ungleichung nicht durch Multiplikation mit $(r+1)^{-1} \cdot (s+1)^{-1}$:

		\[ r\cdot (1+s) \cdot (r+1)^{-1} \cdot (s+1)^{-1} < s \cdot (1+r) \cdot (r+1)^{-1} \cdot (s+1)^{-1} \]
		\[ \dfrac{r}{r+1} < \dfrac{s}{s+1} \]

		Auch hier handelt es sich immer um äquivalente Umformungen. Also wurde die Behauptung unter den Voraussetzungen gezeigt. $\hfill\Box$

	\newpage

	\subsection*{Aufgabe 3}

		\begin{description}
			\item[Voraussetzung:] \hfill \\
				Sei $(K,+,\cdot)$ ein angeordneter Körper mit der Teilmenge $M$. \hfill 
			\item[Behauptung:] \hfill \\
				$-M$ ist nach unten beschränkt $\Leftrightarrow$ $M$ ist nach oben beschränkt \hfill 
			\item[Beweis:] 
		\end{description}

		\underline{$M$ ist nach oben beschränkt $\Rightarrow$ $-M$ ist nach unten beschränkt :} \\

		Es gilt:

		\[ -M = \{-m \ | \ m\in M \} \]

		Für eine obere Schranke $s \in K$ von $M$ gilt für alle $m \in M$

		\[ m \leq s \]

		Multipliziert man diese Ungleichung mit $-1$, dann muss sich ihr Relationszeichen umkehren, damit sie weiterhin gilt.

		\[ (-1)\cdot m \geq (-1)\cdot s \]
		\[ -m \geq -s \]

		Alle Elemente $-m$ für $m \in M$ befinden sich in $-M$. Damit gilt also für alle $m' \in -M$

		\[ m' \geq -s \]

		Für eine untere Schranke $s' \in K$ von $-M$ gilt für alle $m' \in -M$

		\[ m' \geq s' \]

		Da $-s \in K$, ist also $-s$ für $-M$ eine untere Schranke, wenn man $s'=-s$ setzt. Besitzt also $M$ eine obere Schranke $s \in K$, so muss $-M$ eine untere Schranke $-s$ besitzen.\\


		\underline{$-M$ ist nach unten beschränkt $\Rightarrow$ $M$ ist nach oben beschränkt : (Beweis analog)} \\

		Für eine untere Schranke $s' \in K$ von $-M$ gilt für alle $m' \in -M$

		\[ m' \geq s' \]

		Auch hier multipliziert man wieder mit $-1$ und erhält:

		\[ -m' \leq -s' \]

		Schreibt man diese Gleichung für die Elemente aus $M$ auf, folgt für alle $m \in M$:

		\[ -(-m) \leq -s' \]
		\[ m \leq -s' \]

		Für eine obere Schranke $s \in K$ von $M$ gilt wieder für alle $m \in M$

		\[ m \leq s \]

		Setzt man nun $s=-s'$, so erkennt man, dass $-s' \in K$ eine obere Schranke für $M$ bildet. Besitzt also $-M$ eine untere Schranke $s' \in K$, so muss $M$ eine obere Schranke $-s'$ besitzen. \\

		Damit wurden beide Richtungen der Äquivalenzaussage gezeigt. $\hfill\Box$ \\



		\begin{description}
			\item[Voraussetzung:] \hfill \\
				Sei $(K,+,\cdot)$ ein angeordneter Körper mit der Teilmenge $M$. \hfill 
			\item[Behauptung:] \hfill \\
				$sup(M)$ existiert $\Leftrightarrow$ $inf(-M)$ existiert \hfill \\
				Es soll dann gelten: $-sup(M) = inf(-M)$ \hfill 
			\item[Beweis:] 
		\end{description}

		\underline{$sup(M)$ existiert $\Rightarrow$ $inf(-M)$ existiert :} \\

		Jedes Supremum einer Menge ist auch eine obere Schranke dieser Menge. Existiert also $sup(M) \in K$, so gibt es nach obigem Beweis eine untere Schranke $-sup(M)$ für die Menge $-M$.
		Weiterhin gilt für alle oberen Schranken $s \in K$ von $M$:

		\[ sup(M) \leq s \]

		Für jede obere Schranke $s \in K$ von $M$ muss $-M$ genau eine untere Schranke $-s$ besitzen, da für jede untere Schranke in $-M$ auch eine obere Schranke in $M$ existieren muss.
		Multipliziert man also die Ungleichung mit $-1$, so folgt:

		\[ -sup(M) \geq -s \]

		Da $-s$ auch eine untere Schranke von $-M$ ist, gilt für alle unteren Schranken $s' \in K$ von $-M$

		\[ -sup(M) \geq s' \]

		Für ein Infimum von $-M$ gilt dann:

		\[ inf(-M) \geq s' \]

		Damit muss nach der Definition eines Infimums $-sup(M)$ ein Infimum von $-M$ sein. Da sowohl Infimum als auch Supremum eindeutig sind, muss also

		\[ -sup(M) = inf(-M) \]

		gelten. Da $sup(M)$ existiert, muss demnach auch, da es sich bei $K$ um einen Körper handelt, $-sup(M)$ existieren. Also existiert auch $inf(-M)$, da diese Werte gleich sind. \\



		\underline{$inf(-M)$ existiert $\Rightarrow$ $sup(M)$ existiert : (Beweis analog)} \\

		Jedes Infimum einer Menge ist auch eine untere Schranke dieser Menge. Existiert also $inf(-M) \in K$, so gibt es nach obigem Beweis eine obere Schranke $-inf(-M)$ für die Menge $M$.
		Weiterhin gilt für alle unteren Schranken $s' \in K$ von $-M$:

		\[ inf(-M) \geq s' \]

		Für jede untere Schranke $s' \in K$ von $-M$ muss $M$ genau eine obere Schranke $-s'$ besitzen.
		Multipliziert man also die Ungleichung mit $-1$, so folgt:

		\[ -inf(-M) \leq -s' \]

		Da $-s'$ auch eine obere Schranke von $M$ ist, gilt für alle oberen Schranken $s \in K$ von $M$

		\[ -inf(-M) \leq s \]

		Für ein Supremum von $M$ gilt dann:

		\[ sup(M) \leq s \]

		Damit muss nach der Definition eines Supremums $-inf(-M)$ ein Supremum von $M$ sein. Da sowohl Infimum als auch Supremum eindeutig sind, muss also

		\[ sup(M) = -inf(-M) \]
		\[ -sup(M) = inf(-M) \]

		gelten. Da $inf(-M)$ existiert, muss demnach auch, da es sich bei $K$ um einen Körper handelt, $-inf(-M)$ existieren. Also existiert auch $sup(M)$, da diese Werte gleich sind.\\


		Damit wurde die äquivalente Aussage gezeigt. Aus beiden Richtungen folgt dann 

		\[-sup(M) = inf(-M) \]
		 $\hfill\Box$
 

	\newpage

 	\subsection*{Aufgabe 4}

 		\begin{description}
			\item[Voraussetzung:] \hfill \\
				Sei $\mathbb{Q}$ der Körper der rationalen Zahlen. \hfill 
			\item[Behauptung:] \hfill \\
				$\{ q\in \mathbb{Q} \ | \ q^2 \leq 2 \}$ besitzt kein Supremum \hfill
			\item[Beweis:] 
		\end{description}

		Für $\mathbb{Q}$ gilt:

		\[ \mathbb{Q} = \left\{ \dfrac{n}{m} \ \vert \ n \in \mathbb{Z} \ m \in \mathbb{N} \right\} \]

		Sei die Menge $T$ definiert als:

		\[ T := \{ q\in \mathbb{Q} \ | \ q^2 \leq 2 \} \]

		Dann gilt für alle $q \in T$:

		\[ q^2 \leq 2 \]

		Aus den Beweisen der Vorlesung ist bekannt, dass dann  für alle $q \in T$

		\[ q \leq \sqrt{2} \]

		gilt, sofern $q \geq 0$ erfüllt ist. Ist  $q < 0$, so reicht es, diese Ungleichung mit $-1$ zu multiplizieren.
		Wenn also nun $\sqrt{2} \in \mathbb{Q}$ ist, so muss nach der Definition einer oberen Schranke $\sqrt{2}$ eine obere Schranke sein.
		Für jede weitere obere Schranke $s \in \mathbb{Q}$ muss gelten:

		\[ s \geq \sqrt{2} \]

		Da sonst $s \in T$ wäre. Es würde dann für alle $q \in T$ gelten:

		\[ (q \leq s < \sqrt{2}) \Rightarrow (q^2 \leq s^2 < 2) \]

		Damit kann man $s^2$ auch als $2-\epsilon$ für $\epsilon \in \mathbb{Q}$ und $\epsilon > 0$ beschreiben, da es sich bei $\mathbb{Q}$ um einen Körper handelt und dieser abgeschlossen zur Addition ist.
		Damit gilt also automatisch für $a,b \in \mathbb{N}$ und für alle $q \in T$:

		\[ \left(\epsilon = \dfrac{a}{b}\right) \Rightarrow (q \leq 2-\dfrac{a}{b} < 2) \]

		Nun gibt es aber ein $\epsilon' \in \mathbb{Q}$, für welches Folgendes gelten kann:

		\[ \epsilon' = \dfrac{a}{2b} > 0 \]

		Da $a,b \in \mathbb{N}$ muss auch $2b \in \mathbb{N}$ sein.

		\[ \Rightarrow \left( 2- \epsilon' = 2 - \dfrac{a}{2b} < 2 \right) \]

		Damit müsste die Quadratwurzel Element von $T$ sein. Wenn nun $2-\epsilon$ eine obere Schranke bildet, dann gilt:

		\[ (2-\epsilon \geq 2 -\epsilon') \Rightarrow (\epsilon \leq \epsilon') \Rightarrow (\epsilon - \epsilon' \leq 0) \]
		\[ \epsilon - \epsilon' = \dfrac{a}{b} - \dfrac{a}{2b} = \dfrac{2a}{2b} - \dfrac{a}{2b} = \dfrac{a}{2b}\cdot (2-1) = \dfrac{a}{2b} \]

		\[ (a,b \in \mathbb{N}) \Rightarrow (a,b > 0) \Rightarrow \left( \dfrac{a}{2b} > 0 \right) \Rightarrow (\epsilon - \epsilon' > 0) \]

		Dies ist ein Widerspruch zur Annahme, dass $s$ eine obere Schranke ist. Damit kann also keine obere Schranke kleiner als $\sqrt{2}$ sein. $\sqrt{2}$ müsste also das Supremum von $T$ sein. Wenn also $\sqrt{2}$ als Supremum existiert, muss Folgendes für ein $p,q \in \mathbb{N}$, wenn $p$ und $q$ teilerfremd, gelten:

		\[ \sqrt{2} = \dfrac{p}{q} \Rightarrow 2 = \dfrac{p^2}{q^2} \Rightarrow 2q^2 = p^2 \] 

		Damit muss $p^2$ ein Vielfaches von $2$ sein. Da es sich um eine natürliche Zahl handelt, muss damit auch $p$ ein Vielfaches von $2$ darstellen. Man kann also ein $k \in \mathbb{N}$ finden, für welches $p = 2k$ ist.

		\[ 2q^2 = (2k)^2 \Rightarrow 2q^2 = 4k^2 \Rightarrow q^2 = 2k^2 \]

		Damit müsste also auch $q^2$ und damit auch $q$ ein Vielfaches von $2$ sein. Dies ist allerdings ein Widerspruch dazu, dass $p,q$ teilerfremd sind. Es gibt also keine Darstellung für $\sqrt{2}$ in den rationalen Zahlen. Damit existiert das Supremum von $T$ mit $\sqrt{2}$ nicht.

		Eine andere Möglichkeit wäre, dass man sich eine andere obere Schranke sucht. Das Quadarat dieser oberen Schranke könnte man wieder durch $\epsilon + 2$ beschreiben. Handelt es sich um ein Supremum, müssen also alle anderen oberen Schranken größer als $\epsilon + 2$ sein. Nun können wir analog zur obigen Betrachtung ein $\epsilon' + 2$ finden, welches kleiner dieser oberen Schranke ist und dennoch nicht in $T$ ist. Damit gäbe ES auf die gleiche Weise einen Widerspruch, dass es außer $\sqrt{2}$ noch ein weiteres Supremum (ob nun kleiner oder größer) geben kann. Damit ist allgemein gezeigt, da $\sqrt{2}$ kein Element von $\mathbb{Q}$ ist , dass das Supremum von $T$ nicht existiert.$\hfill\Box$




	\newpage

	\subsection*{Zusatzaufgabe}

		\begin{description}
			\item[Voraussetzung:] \hfill \\
				Seien $n \in \mathbb{N}$ und $a_1,...,a_n,b_1,...,b_n \in \mathbb{R}$ gegeben. Sei $z_m := \sum_{k=1}^m a_k$ für $m \in \mathbb{N}$ mit $1\leq m\leq n$.\hfill 
			\item[Behauptung:] \hfill \\
				Dann gilt: $\sum_{k=1}^n a_kb_k = z_nb_n + \sum_{k=1}^{n-1} z_n(b_k - b_{k+1})$ \hfill
			\item[Beweis:] 
		\end{description}

		Seien die Variablen und Koeffizienten wie in der Voraussetzung definiert. Dann gilt:

		\[ z_nb_n + \sum_{k=1}^{n-1} z_n(b_k - b_{k+1}) =  b_n\cdot\sum_{k=1}^n a_k  +  \sum_{k=1}^{n-1}\left( (b_k - b_{k+1})\cdot \sum_{i=1}^k a_i \right) \]

		Durch Ausmultiplizieren im hinteren Summenzeichen erhält man:

		\[ = b_n\cdot\sum_{k=1}^n a_k  +  \sum_{k=1}^{n-1}\left( b_k\cdot \sum_{i=1}^k a_i -b_{k+1}\cdot \sum_{i=1}^k a_i \right) \]
		\[ = b_n\cdot\sum_{k=1}^n a_k  +  \sum_{k=1}^{n-1}\left( b_k\cdot \sum_{i=1}^k a_i \right) - \sum_{k=1}^{n-1} \left(b_{k+1}\cdot \sum_{i=1}^k a_i \right) \]

		Aus dem mittleren Summenzeichen soll nun das erste Element herausgezogen werden. Das gleiche soll für das letzte Element des rechten Summenzeichens getan werden.

		\[ = b_n\cdot\sum_{k=1}^n a_k  + b_1\cdot\sum_{i=1}^1a_i + \sum_{k=2}^{n-1}\left( b_k\cdot \sum_{i=1}^k a_i \right) - b_n\cdot\sum_{i=1}^{n-1}a_i - \sum_{k=1}^{n-2} \left(b_{k+1}\cdot \sum_{i=1}^k a_i \right) \]
		\[ = \left(b_n\cdot\sum_{k=1}^n a_k - b_n\cdot\sum_{i=1}^{n-1}a_i \right) + a_1b_1 + \sum_{k=2}^{n-1}\left( b_k\cdot \sum_{i=1}^k a_i \right) - \sum_{k=1}^{n-2} \left(b_{k+1}\cdot \sum_{i=1}^k a_i \right) \]
		\[ = a_nb_n + a_1b_1 + \sum_{k=2}^{n-1}\left( b_k\cdot \sum_{i=1}^k a_i \right) - \sum_{k=1}^{n-2} \left(b_{k+1}\cdot \sum_{i=1}^k a_i \right) \]

		Verschiebt man den Index des rechten Summenzeichens um $1$, folgt:

		\[ = a_nb_n + a_1b_1 + \sum_{k=2}^{n-1}\left( b_k\cdot \sum_{i=1}^k a_i \right) - \sum_{k=2}^{n-1} \left(b_{k}\cdot \sum_{i=1}^{k-1} a_i \right) \]
		\[ = a_nb_n + a_1b_1 + \sum_{k=2}^{n-1}\left( b_k\cdot \left(\sum_{i=1}^k a_i  - \sum_{i=1}^{k-1} a_i \right) \right) \]
		\[ = a_nb_n + a_1b_1 + \sum_{k=2}^{n-1} a_kb_k \]
		\[ = \sum_{k=1}^n a_kb_k \]

		Durch äquivalente Umformungen wurde damit die Gleichheit gezeigt. $\hfill\Box$

\end{document}
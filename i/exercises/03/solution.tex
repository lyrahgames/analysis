\documentclass[11pt, a4paper]{article}

\usepackage[utf8]{inputenc}

\usepackage{amsmath}
\usepackage{amssymb}
\usepackage{float}

\usepackage{geometry}
\geometry{a4paper, lmargin=30mm, rmargin=20mm, tmargin=30mm, bmargin=20mm}

\setlength{\parindent}{0mm}

\begin{document}

	\begin{center} \section*{Analysis I - Übungsserie 3} \end{center}

	Übungsgruppe: Jonas Franke \\


	Nina Held: 144753 \\
	Clemens Anschütz: 146390 \\
	Markus Pawellek: 144645 \\


	\subsection*{Aufgabe 1}

		\begin{description}
			\item[Voraussetzung:] \hfill \\
				$(N,e,\nu)$ genüge den Peano Axiomen. Für $n \in N$ soll $A_e = \{e\}$ und $A_{\nu(n)} = A_n \cup \{\nu(n)\}$. \\
				Für alle $n \in N$ sind Aussagen $B(n)$ gegeben. Dabei soll $B(e)$ wahr sein und gelten, dass wenn für alle $k \in A_n$  $B(k)$ gilt, auch $B(\nu(n))$ wahr ist. \hfill 
			\item[Behauptung:] \hfill \\
				Für alle $n \in N$ ist $B(n)$ wahr. \hfill 
			\item[Beweis:] 
		\end{description}

		Sei die Menge $L$ definiert als:

		\[ L := \{ n \in N \ | \ \text{für alle } k \in A_n \text{ gilt } B(k) \text{ ist wahr.} \} \]

		Dann gilt wegen $A_e = \{e\}$, dass:

		\[ B(e) \text{ ist wahr. } \Rightarrow \text{für alle } k \in A_e \text{ gilt } B(k) \text{ ist wahr. } \Rightarrow (e \in L) \]

		Nimmt man nun an, dass $n \in L$, folgt aus den Voraussetzungen:

		\[ (n \in L) \Rightarrow \text{für alle } k \in A_n \text{ gilt } B(k) \text{ ist wahr.} \Rightarrow B(\nu(n)) \text{ ist wahr.} \]

		Für $A_{\nu(n)}$ gilt:

		\[ A_{\nu(n)} = A_n \cup \{\nu(n)\} \]

		Da $n \in L$ gilt, ist für alle $k \in A_n$ $B(k)$ wahr. Jetzt gerade wurde gezeigt, dass dann auch $B(\nu(n))$ wahr ist.
		Damit ist $B$ auch für alle Elemente aus $A_{\nu(n)}$ wahr.

		\[ \Rightarrow \text{für alle } k \in A_{\nu(n)} \text{ gilt } B(k) \text{ ist wahr.} \Rightarrow (\nu(n) \in L) \]

		Damit folgt aufgrund des zweiten Peanoaxioms:

		\[ ((e \in L) \wedge ((n \in L) \Rightarrow (\nu(n) \in L))) \Rightarrow (L=N) \]

		$B(n)$ muss damit für alle $n \in N$ gelten, wenn die Voraussetzungen erfüllt sind. $\hfill \Box$

		\newpage


	\subsection*{Aufgabe 2}

		\begin{description}
			\item[Voraussetzung:] \hfill \\
				Sei $(B,e,\nu)$ gegeben mit bijektiver Abbildung $\nu:B \longrightarrow B$ und einer Totalordnung $\leq$ mit $n \leq \nu(n)$ für alle $n \in B$. \hfill 

			\item[Behauptung:] \hfill 
				$(B,\leq)$ ist wohlgeordnet $\Leftrightarrow (B,e,\nu)$ erfüllt das zweite Peanoaxiom. \hfill \\

			\item[Beweis:] 
		\end{description}

		Beweis: $(B,e,\nu)$ erfüllt das zweite Peanoaxiom $\Rightarrow (B,\leq)$ ist wohlgeordnet.\\

		Annahme: Für eine beliebige Teilmenge T mit $\varnothing \neq T \subseteq B$ existiert kein kleinstes Element.\\

		Sei die Menge $L$ definiert als:

		\[ L := B \setminus T \]


		Für ein kleinstes Element $n$ einer beliebigen Menge $M$ muss $n \leq x$ für alle $x \in M$ gelten. Da $n \leq \nu(n)$ darf es also kein Element $k \in M$ geben, für welches $\nu(k) = n$ gilt, weil sonst auch $k \leq n$ gelten würde. Betrachtet man diese Implikation für $B$, so folgt, dass für ein kleinstes Element $n \in B$ für alle $x \in B$ $\nu(x) \neq n$ sein muss. Da $\nu(n)$ eine bijektive Abbildung auf $B\setminus \{e\}$ ist, kann nur $e$ diese Kriterien erfüllen. Damit ist $e$ das kleinste Element von $B$. \\

		Aus diesem Grund kann $e$ kein Element von $T$ sein, da $e$ bezüglich einer Teilmenge von $B$ immer ein kleinstes Element sein muss.

		\[ \neg(e \in T) \Rightarrow (e \in L) \]

		Nimmt man nun an, dass $n \in L$, folgt:

		\[ (n \in L) \Rightarrow \neg(n \in T) \]

		Damit wäre $n$ bezüglich $T$ ein kleinstes Element. Da $n$ aber nun nicht mehr in $T$ vorhanden sein kann, gibt es kein weiteres Element $k \in T$ für welches $\nu(k) = \nu(n)$ ist, da $\nu$ bijektiv ist. Dies kann grundsätzlich für mehrere Elemente in $T$ gelten, aber muss auch zwangsläufig für das kleinste Element gelten, welches vorhanden sein muss, solange dies gilt. Aufgrund der gegebenen Totalordnung lassen sich diese Elemente immer miteinander vergleichen.
		Damit muss es also unter diesen Elementen immer ein kleinstes Element geben, welches nicht in $T$ sein darf. Dies impliziert, dass alle Elemente, welche durch $\nu(k)$ mit $k \in T$ nicht dargestellt werden können, entfernt werden müssen (Durch Entfernung des Kleinsten taucht wieder ein kleinstes Element auf.)

		\[ \Rightarrow \neg(\nu(n) \in T) \Rightarrow (\nu(n) \in L) \] 

		Nach dem zweiten Peanoaxiom ergibt sich, dass $B = L$. Da $T$ das Komplement der Menge $L$ ist, muss $T$ eine leere Menge sein. Dies ist ein Widerspruch zur Annahme, dass $T$ keine leere Menge ist. Damit muss jede Teilmenge von $B$ ein kleinstes Element enthalten. Damit folgt aus dem zweiten Peanoaxiom, dass $B$ wohlgeordnet ist. \\

		\newpage

		Beweis: $(B,\leq)$ ist wohlgeordnet $\Rightarrow (B,e,\nu)$ erfüllt das zweite Peanoaxiom. \\

		Sei die Menge $M$ definiert als:

		\[ M \subseteq B \]
		\[ e \in M \]
		\[ (n \in M) \Rightarrow (\nu(n) \in M) \]

		Zeigt man, dass $M=B$, dann muss das zweite Peanoaxiom gelten. Weiterhin soll $L$ definiert sein als:

		\[ L := B\setminus M \]

		Die Menge $L$ bezeichnet also die Elemente von $B$, welche durch die Voraussetzungen des Peanoaxioms ausgeschlossen werden. Damit ist $L \subseteq B$ und es muss aufgrund der Wohlordnung gelten, dass es ein kleinstes Element $x$ in $L$ geben muss. Wegen $x \neq e$ (da $e \in M$)kann $x$ auch durch $\nu(k)$ für ein $k \in B$ beschrieben werden. Damit kann $k$ also nicht Element von $L$ sein, da dann wegen $k \leq \nu(k)$ noch ein kleineres Element in $L$ existieren würde.

		\[ \Rightarrow (\nu(k) \in L) \Rightarrow \neg(k \in L) \Rightarrow (k \in M) \]

		Aufgrund der Voraussetzung für $M$ gilt dann:

		\[ \Rightarrow (\nu(k) \in M) \]

		Da aber $L$ als Komplement zu $M$ definiert wurde und damit die beiden Mengen disjunkt sind, liegt ein Widerspruch für jedes Element von $L$ vor.

		\[ \Rightarrow (L = \varnothing) \Rightarrow (M = B) \]

		Damit gilt das zweite Peanoaxiom, wenn B wohlgeordnet ist.\\

		Der Beweis wurde für beide Richtungen erbracht. $\hfill\Box$

		\newpage


	\subsection*{Aufgabe 3}

		\begin{description}
			\item[Voraussetzung:] \hfill \\
				$(K,+,\cdot)$ ist ein Körper. \hfill 

			\item[Behauptung:] \hfill \\
				$(-x)\cdot(-y) = x\cdot y$ gilt. \hfill 

			\item[Beweis:] 
		\end{description}

		In der Vorlesungen wurde folgende Proportion für $x \in K$ bewiesen:

		\[ -x = (-1) \cdot x \]

		Damit gilt also:

		\[ (-x)\cdot(-y) = (-1) \cdot x \cdot (-1) \cdot y \]

		Bezüglich der Multiplikation gilt nach den Körperaxiomen das Kommutativgesetz:

		\[ (-x)\cdot(-y) = (-1) \cdot (-1) \cdot x \cdot  y \]

		Wenn die Behauptung gilt, muss durch Addition mit dem Inversen zur Addition das neutrale Element der Addition (hier $0$) herauskommen.

		\[ x\cdot y + (-(x\cdot y)) = 0 \]

		Setzt man die Behauptung und die Proposition ein, folgt:

		\[ (-1) \cdot (-1) \cdot x \cdot y + (-1)\cdot(x\cdot y) = 0 \]

		Nun wendet man das Distributivgesetz an, welches nach den Körperaxiomen gilt:

		\[ (-1)\cdot x\cdot y \cdot ((-1)+1) = 0 \]

		$-1$ ist das Inverse bezüglich der Addition zu $1$. Damit gilt nach den Axiomen für das inverse Element $(-1) + 1 = 0$.

		\[ (-1)\cdot x\cdot y \cdot 0 = 0 \]

		In der Vorlesung wurde gezeigt, dass folgende Proposition für $x \in K$ gilt:

		\[ x\cdot 0 = 0 \]

		$(-1)\cdot x\cdot y$ ist ein Element von $K$ und kann damit mit einem $z \in K$ gleich gesetzt werden.

		\[ z = (-1)\cdot x\cdot y \]

		Durch Einsetzen folgt:

		\[ z \cdot 0 = 0 \]

		Dies ist eine wahre Aussage, wenn man vorherige Proposition anwendet. Aufgrund der Eindeutigkeit eines inversen Elementes, folgt also, dass $(-x)\cdot(-y)$ das Inverse bezüglich der Addition zu $-(x\cdot y)$ ist. Dies gilt aber auch für $x\cdot y$.

		\[ \Rightarrow (-x)\cdot(-y) = x\cdot y \] 
		$\hfill\Box$

		\newpage


	\subsection*{Aufgabe 4}


		Folgende Operationen sollen für $\mathbb{F}_3$ gelten: \\


		\begin{figure}[H]
		\center

		\begin{tabular}[c]{cc}
			\begin{tabular}[c]{|c||c|c|c|}
				\hline
				$+$ & $0$ & $1$ & $2$ \\
				\hline \hline
				$0$ & $0$ & $1$ & $2$ \\
				\hline
				$1$ & $1$ & $2$ & $0$ \\
				\hline
				$2$ & $2$ & $0$ & $1$ \\
				\hline
			\end{tabular}
		
			&

			\begin{tabular}[c]{|c||c|c|c|}
				\hline
				$\cdot$ & $0$ & $1$ & $2$ \\
				\hline \hline
				$0$ & $0$ & $0$ & $0$ \\
				\hline
				$1$ & $0$ & $1$ & $2$ \\
				\hline
				$2$ & $0$ & $2$ & $1$ \\
				\hline
			\end{tabular} \\

		\end{tabular}

		\end{figure}

		(In der Tat ist dies auch die einzige Möglichkeit der Operationen, für die $\mathbb{F}_3$ ein Körper ist.)\\

		Dabei soll $0$ das neutrale Element bezüglich der Addition sein. Und $1$ das neutrale Element bezüglich der Multiplikation. Aus den Tabellen wird ersichtlich, dass die Axiome für die neutralen Elemente erfüllt sind. 
		Die inversen Elemente bezüglich der Addition und der Multiplikation sind ebenfalls gegeben mit $-1 = 2$, $-2 = 1$ und $2^{-1} = 2$ (für die neutralen Elemente muss dies nicht gezeigt werden, da es sich automatisch aus den Axiomen für sie ergibt).
		Außerdem sieht man anhand der Tabellen, dass beide Operationen kommutativ sind, da durch Austausch von Zeilen und Spalten die Tabellen wieder auf sich selbst abgebildet werden.


		Um das Assoziativitätsgesetz für die Addition zu zeigen, muss $0$ nicht betrachtet werden, da durch Addition mit ihr immer das gleiche wie vorher herauskommt.

		\[ 1+(1+2) = (1+1)+2 \]
		\[ 1+0 = 2+2 \]
		\[ 1 = 1 \]

		Damit ergibt sich eine wahre Aussage. Aufgrund des Kommutativitätsgesetzes können diese drei Zahlen beliebig untereinander vertauscht werden. Damit ist es also auch gezeigt für $1+(2+1) = (1+2)+1$ und $2+(1+1) = (2+1)+1$.

		\[ 1+(2+2) = (1+2)+2 \]
		\[ 1+1 = 0+2 \]
		\[ 2 = 2 \]

		Auch hier ergibt sich eine wahre Aussage. Aufgrund des Kommutativitätsgesetzes können auch diese drei Zahlen beliebig untereinander vertauscht werden. Damit ist es also auch gezeigt für $2+(1+2) = (2+1)+2$ und $2+(2+1) = (2+2)+1$.

		Um das Assoziativitätsgesetz der Multiplikation zu zeigen, muss $0$ nicht betrachte werden, da durch Multiplikation mit $0$ immer alles $0$ wird. Damit gilt für $0$ automatisch die Assoziativität. Auch für $1$ muss es nicht gezeigt werden, da durch Multiplikation mit $1$ immer das gleiche wie vorher herauskommt. Übrig bleibt damit nur die $2$. Diese ist zu sich selbst kommutativ und assoziativ.

		Für das Distributivgesetz muss $0$ aus den beiden oben genannten Gründen nicht beachtet werden. Für $a,b \in K$ gilt:

		\[ 1\cdot (a+b) = 1\cdot a + 1\cdot b \]
		\[ a+b = a+b \]

		Wahre Aussage.
		
		\[ 2\cdot (1+1) = 2\cdot 1 + 2\cdot 1 \]
		\[ 2\cdot 2 = 2 + 2 \]
		\[ 1 = 1 \]

		Wahre Aussage.

		\[ 2\cdot (1+2) = 2\cdot 1 + 2\cdot 2 \]
		\[ 2\cdot 0 = 2+1 \]
		\[ 0 = 0 \]

		Wahre Aussage. Aufgrund der Kommutativität sind $1$ und $2$ in den Klammern vertauschbar.

		\[ 2\cdot (2+2) = 2\cdot 2 + 2\cdot 2 \]
		\[ 2\cdot 1 = 1+1 \]
		\[ 2 = 2 \]

		Wahre Aussage. Damit sind alle Möglichkeiten für das Distributivgesetz gezeigt. Es folgt, dass $\mathbb{F}_3$ ein Körper ist.
		Für einen angeordneten Körper muss jedes Element in nur einer der Mengen $K^+$, $\{0\}$ oder $\{-x \ | x \in K^+ \}$ vorkommen.
		Außerdem wurde in der Vorlesung gezeigt, dass $1 \in K^+$ gilt. Des Weiteren gilt:

		\[ (x,y \in K^+) \Rightarrow (x+y \in K^+) \]
		\[ (x,y \in K^+) \Rightarrow (x\cdot y \in K^+) \]

		Aus diesen Axiomen folgt für $\mathbb{F}_3$:

		\[ (1\in K^+) \Rightarrow (1+1 \in K^+) \Rightarrow (2 \in K^+) \Rightarrow (2+1 \in K^+) \Rightarrow (0 \in K^+) \]

		Damit müsste $0$ sowohl Element von $K^+$ als auch $\{0\}$ sein. Damit würde $0$ als Element in beiden Mengen vorkommen. Dies widerspricht allerdings den Axiomen für einen angeordneten Körper (betrachtete Mengen müssen disjunkt sein). Damit kann $\mathbb{F}_3$ nicht angeordnet werden.
 

\end{document}
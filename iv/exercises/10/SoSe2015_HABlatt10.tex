%\documentclass[12pt,a4paper]{article}
%\usepackage{a4kopka}
\documentclass[12pt,a4paper]{article}
\usepackage{ngerman}
%\usepackage{fleqn}
\usepackage{array}
\usepackage[latin1]{inputenc}
\usepackage{amssymb}
\usepackage{amsmath}
\usepackage{amsfonts}
%\usepackage{mathbbol}
\usepackage{amstext}
%\usepackage{fancyhdr}
\usepackage{relsize}
\usepackage{epic}
\usepackage{graphics}
\usepackage{graphicx}
\usepackage[T1]{fontenc}
\usepackage{a4wide}
%\usepackage{eufrak}
\usepackage{makeidx}%\usepackage{german}
\usepackage{textcomp}

\newcommand{\D}{\displaystyle}
%%%%%%%%%%%%%%%%%%%%

\newcommand{\IR}{{\mathbb{R}}}
\newcommand{\IN}{{\mathbb{N}}}
\newcommand{\IZ}{{\mathbb{Z}}}
\newcommand{\IF}{{\mathbb{F}}}
\newcommand{\IK}{{\mathbb{K}}}
\newcommand{\IQ}{{\mathbb{Q}}}
\newcommand{\IC}{{\mathbb{C}}}
\newcommand{\IP}{{\mathbb{P}}}
\newcommand{\IE}{{\mathbb{E}}}

\newcommand{\ep}{{\varepsilon}}
\newcommand{\ph}{{\varphi}}
\newcommand{\thet}{{\vartheta}}
\newcommand{\rh}{{\varrho}}
\newcommand{\de}{{\delta}}
\newcommand{\la}{{\lambda}}
\newcommand{\Om}{{\Omega}}
\newcommand{\al}{{\alpha}}
\newcommand{\be}{{\beta}}
\newcommand{\ga}{{\gamma}}
\newcommand{\om}{{\omega}}
\newcommand{\La}{{\Lambda}}
\newcommand{\Ga}{{\Gamma}}
\newcommand{\De}{{\Delta}}

\newcommand{\foh}{{\mathfrak{h}}}

\newcommand{\nach}{{\rightarrow}}
\newcommand{\Nach}{{\,\rightarrow\,}}
\newcommand{\Fou}{{\mathcal{F}}}
\newcommand{\sk}{{\,|\,}}


%\newcommand{\cosh}{\mbox{cosh}}
\newtheorem{theorem}{Theorem}[section]
\newtheorem{lemma}[theorem]{Lemma}
\newtheorem{proposition}[theorem]{Proposition}
\newtheorem{corollary}[theorem]{Corollary}
\newtheorem{remarks}[theorem]{Remarks}
%%%%%%%%%%%%%%%%%%%%

\addtolength{\voffset}{-15pt} \addtolength{\textheight}{-10pt}
%\addtolength{\textheight}{-11pt} \addtolength{\headsep}{6pt}

\setlength{\parskip}{10pt plus 2pt minus 1pt}
\setlength{\parindent}{0pt}
\newcommand{\wk}{\mbox{$\,<$\hspace{-5pt}\footnotesize )$\,$}}
\thispagestyle{empty} \addtolength{\voffset}{15pt}



\begin{document}

\rule{\textwidth}{0.3pt}
\begin{center}
\textbf{\large H\"ohere Analysis I}
\end{center}
\textbf{Sommersemester 2015 \hfill Prof. Dr. D. Lenz}

\rule{\textwidth}{0.3pt}

%%%%%%%%%%%%%%%%%%%%%%%%%%%%%%%%%%%%%%%%%%%%%%%%%%%%
%
% Hier geht's los
%
%%%%%%%%%%%%%%%%%%%%%%%%%%%%%%%%%%%%%%%%%%%%%%%%%%%%%
\textbf{Blatt 10}\hfill % Nr des Blatts
\textbf{Abgabe Dienstag 07.07.2015}

%%%%%%%%%%%%%%%%%%%%%%%%%%%%%%%%%%%%%%%%%%%%%%%%%%%%
\begin{itemize}
\item[(1)]  Sei $(X,d)$ ein metrischer Raum. Der Vektorraum $C_0(X)$ ist definiert als die Menge aller stetigen Funktionen $f:X\to\IC$, sodass f\"ur jedes $\varepsilon>0$ eine kompakte Teilmenge $K_\varepsilon\subset X$ existiert mit $|f(x)|\leq\varepsilon$ f\"ur alle $x\not\in K_\varepsilon$. Zeigen Sie die folgenden Aussagen.
\begin{itemize}
\item[(a)] F\"ur $X=\IR$ ausgestattet mit der euklidischen Metrik gilt die Gleichheit 
$$C_0(\IR)=\{f:\IR\to\IC\;|\; f\text{ stetig mit } \lim\limits_{x\to\pm\infty} f(x)=0\}.$$
\item[(b)] F\"ur $X=\IN$ mit der diskreten Metrik gilt die Gleichheit 
$$C_0(\IN)=\{f:\IN\to\IC\;|\; \lim\limits_{n\to\infty} f(n)=0\}.$$
\end{itemize}

\item[(2)] Sei $\nu$ ein positives Ma\3 und
$\mu$ ein komplexes Ma\3 auf dem Ma\ss raum $(X,\mathcal{A})$. Zeigen Sie, dass folgende Aussagen \"aquivalent sind.

\begin{itemize}
\item[(i)] Es ist $\mu$ absolut stetig bez\"uglich $\nu$.
\item[(ii)] Zu jedem $\varepsilon >0$ existiert ein $\delta>0$ mit $|\mu (E)|< \varepsilon$ falls $E\in\mathcal{A}$ und $\nu (E) < \delta$.
\end{itemize}

\item[(3)] Sei $\nu$ ein positives Ma\3\, auf dem Ma\ss raum $(X,\mathcal{A})$, $h\in\mathcal{L}^1 (X,\nu)$ und $S\subset \mathbb{C}$ eine abgeschlossene Teilmenge mit
$$\frac{1}{\nu (E)} \int_E h d\nu \in S$$
f\"ur alle me\3baren $E$ mit $\nu (E) >0$. Zeigen Sie, dass $h$ fast sicher nur Werte in $S$ annimmt.

\item[(4)] Ein lineares Funktional $\varphi: \ell^\infty(\IN) \to \IR$ hei\ss t Banachlimes, wenn $\varphi$ folgende drei Eigenschaften erf\"ullt:
\begin{itemize}
\item[(i)] Es gilt $\varphi \circ S = \varphi$ f\"ur den Linksshift $S(x_1,x_2,\dots):= (x_2, x_3, \dots)$,
\item[(ii)] Sind alle $x_k \geq 0$, so ist $\varphi(x)\geq 0$,
\item[(iii)] F\"ur die Folge $e = (1,1,\dots)$ ist $\varphi(e)=1$.
\end{itemize}
Zeigen Sie:
\begin{itemize}
\item[(a)] Ist $\varphi : \ell^\infty(\IN) \to \IR$ ein Banachlimes, so gilt:
\begin{itemize}
\item[(1)]F\"ur alle $x=(x_n)\in \ell^\infty(\IN)$ ist $\liminf x_n \leq \varphi(x)\leq \limsup x_n$.
\item[(2)]$\varphi$ ist stetig mit Norm $\|\varphi\| =1$.
\item[(3)]$\varphi$ ist nicht multiplikativ, d.h. es existieren $x,y \in \ell^\infty(\IN)$ mit $\varphi(x)\varphi(y) \neq \varphi(x\cdot y)$.
\end{itemize}
\item[(b)]Es gibt Banachlimiten.\\
\underline{Hinweis:} Betrachten Sie den Untervektorraum $U:=\{x\in \ell^\infty(\IN):  \lim x_n \mbox{ existiert}\}$ und setzen Sie ein geeignetes lineares Funktional bez\"uglich dem sublinearen Funktional $p(x):= \limsup\frac{1}{n} \sum_{i=0}^{n-1} x_i$ fort.
\end{itemize}

\end{itemize}




\textbf{Zusatzaufgabe.}\\
Sei $(X,\mathcal{A})$ ein Ma\ss raum und $\mu$ ein komplexes Ma\3 auf $\mathcal{A}$. Definiere die Abbildung $|\mu|:\mathcal{A}\to [0,\infty]$ durch

$$|\mu| (E): = \sup \left\{ \left.\sum\limits_{n\in\mathbb{N}} |\mu (E_n)|\;\right|\; (E_n) \text{ Zerlegung von } E\right\},\qquad E\in\mathcal{A}.$$
Zeigen Sie, dass $|\mu|$ eine $\sigma$-additive Abbildung ist.


\end{document}

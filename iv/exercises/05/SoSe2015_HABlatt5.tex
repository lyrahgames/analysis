%\documentclass[12pt,a4paper]{article}
%\usepackage{a4kopka}
\documentclass[12pt,a4paper]{article}
\usepackage{ngerman}
%\usepackage{fleqn}
\usepackage{array}
\usepackage[latin1]{inputenc}
\usepackage{amssymb}
\usepackage{amsmath}
\usepackage{amsfonts}
%\usepackage{mathbbol}
\usepackage{amstext}
%\usepackage{fancyhdr}
\usepackage{relsize}
\usepackage{epic}
\usepackage{graphics}
\usepackage{graphicx}
\usepackage[T1]{fontenc}
\usepackage{a4wide}
%\usepackage{eufrak}
\usepackage{makeidx}%\usepackage{german}
\usepackage{textcomp}

\newcommand{\D}{\displaystyle}
%%%%%%%%%%%%%%%%%%%%

\newcommand{\IR}{{\mathbb{R}}}
\newcommand{\IN}{{\mathbb{N}}}
\newcommand{\IZ}{{\mathbb{Z}}}
\newcommand{\IF}{{\mathbb{F}}}
\newcommand{\IK}{{\mathbb{K}}}
\newcommand{\IQ}{{\mathbb{Q}}}
\newcommand{\IC}{{\mathbb{C}}}
\newcommand{\IP}{{\mathbb{P}}}
\newcommand{\IE}{{\mathbb{E}}}

\newcommand{\ep}{{\varepsilon}}
\newcommand{\ph}{{\varphi}}
\newcommand{\thet}{{\vartheta}}
\newcommand{\rh}{{\varrho}}
\newcommand{\de}{{\delta}}
\newcommand{\la}{{\lambda}}
\newcommand{\Om}{{\Omega}}
\newcommand{\al}{{\alpha}}
\newcommand{\be}{{\beta}}
\newcommand{\ga}{{\gamma}}
\newcommand{\om}{{\omega}}
\newcommand{\La}{{\Lambda}}
\newcommand{\Ga}{{\Gamma}}
\newcommand{\De}{{\Delta}}

\newcommand{\foh}{{\mathfrak{h}}}

\newcommand{\nach}{{\rightarrow}}
\newcommand{\Nach}{{\,\rightarrow\,}}
\newcommand{\Fou}{{\mathcal{F}}}
\newcommand{\sk}{{\,|\,}}


%\newcommand{\cosh}{\mbox{cosh}}
\newtheorem{theorem}{Theorem}[section]
\newtheorem{lemma}[theorem]{Lemma}
\newtheorem{proposition}[theorem]{Proposition}
\newtheorem{corollary}[theorem]{Corollary}
\newtheorem{remarks}[theorem]{Remarks}
%%%%%%%%%%%%%%%%%%%%

\addtolength{\voffset}{-15pt} \addtolength{\textheight}{-10pt}
%\addtolength{\textheight}{-11pt} \addtolength{\headsep}{6pt}

\setlength{\parskip}{10pt plus 2pt minus 1pt}
\setlength{\parindent}{0pt}
\newcommand{\wk}{\mbox{$\,<$\hspace{-5pt}\footnotesize )$\,$}}
\thispagestyle{empty} \addtolength{\voffset}{15pt}



\begin{document}

\rule{\textwidth}{0.3pt}
\begin{center}
\textbf{\large H\"ohere Analysis I}
\end{center}
\textbf{Sommersemester 2015 \hfill Prof. Dr. D. Lenz}

\rule{\textwidth}{0.3pt}

%%%%%%%%%%%%%%%%%%%%%%%%%%%%%%%%%%%%%%%%%%%%%%%%%%%%
%
% Hier geht's los
%
%%%%%%%%%%%%%%%%%%%%%%%%%%%%%%%%%%%%%%%%%%%%%%%%%%%%%
\textbf{Pfingstzettel}\hfill % Nr des Blatts
\textbf{Abgabe Dienstag 26.05.2015}

%%%%%%%%%%%%%%%%%%%%%%%%%%%%%%%%%%%%%%%%%%%%%%%%%%%%
\begin{itemize}
\item[(1)] Sei $(H,\langle\cdot,\cdot \rangle)$ ein Hilbertraum. Eine Folge $(x_n) \subseteq H$ hei\ss t schwach konvergent gegen $x\in H$, wenn f\"ur alle $y \in H$ gilt
%
$$\lim_{n \to \infty}\langle x_n ,y \rangle = \langle x ,y \rangle.$$
%
Zeigen Sie: Eine Folge $(x_n) \subseteq H$ konvergiert genau dann gegen $x$, wenn sie schwach gegen $x$ konvergiert und $\|x_n\| \to \|x\|$ f\"ur $n\to \infty$ gilt. 

\item[(2)] Zeigen Sie, dass eine Norm auf einem reellen Vektorraum genau dann von einem Skalarprodukt erzeugt wird, wenn die Norm die Parallelogrammidentit�t erf�llt.

\item[(3)] F�r eine Funktion $f:\mathbb{N}\to\mathbb{C}$ sei der Tr�ger $supp(f)$ definiert durch die Menge aller Elemente $n\in\mathbb{N}$, sodass $f(n)\neq 0$. Definiere den Vektorraum 
$$
c_c:=\{f:\mathbb{N}\to\mathbb{C}\; :\; supp(f)\subset\mathbb{N} \text{ endlich}\}
$$ 
der endlich getragenen Funktionen auf $\mathbb{N}$ ausgestattet mit dem Skalarprodukt 
$$
\langle f, g \rangle:=\sum\limits_{n\in\mathbb{N}} f(n)\, g(n),\qquad f,g\in c_c.
$$
Zeigen Sie, dass es einen abgeschlossenen Unterraum $U\subset c_c$ mit $U\neq c_c$ gibt, sodass $U^\perp = \{0\}$.

\item[(2)] Seien $m :\IN \to (0,\infty)$ und $1 \leq p < \infty$ gegeben. Wir definieren die gewichteten Folgenr\"aume
%
$$\ell^p(\IN,m) = \{x: \IN \to \IC\, |\, \sum_{n = 1}^\infty |x(n)|^p m(n) < \infty \}$$
%  
und normieren diese mittels
%
$$\|x\|_{p,m} = \left(\sum_{n = 1}^\infty |x(n)|^p m(n)\right)^{1/p}.$$
%
Desweiteren sei $I = \inf_{n \in \IN} m(n) \text{ und } S = \sum_{n=1}^\infty m(n).$ Zeigen Sie, dass f\"ur $1\leq p < q < \infty$ die folgenden Aussagen gelten:
\begin{itemize}
\item[(a)] Falls $I >0$, so gilt $\ell^p(\IN,m) \subseteq \ell^q(\IN,m)$, sowie
%
$$\sup\{\|x\|_{q,m} \,|\, x\in \ell^p(\IN,m), \|x\|_{p,m}\leq 1 \} = I^{\frac{1}{q} - \frac{1}{p}}. $$
% 
\item[(b)] Falls $S < \infty$, so gilt $\ell^q(\IN,m) \subseteq \ell^p(\IN,m)$, sowie
%
$$\sup\{\|x\|_{p,m} \,|\, x\in \ell^q(\IN,m), \|x\|_{q,m}\leq 1 \} = S^{\frac{1}{p} - \frac{1}{q}}. $$
\item[(c)] Ist $I = 0$ bzw. $S = \infty$, so gelten die in (a) bzw. (b) gegebenen Inklusionen nicht.
\end{itemize}

\end{itemize}


\end{document}

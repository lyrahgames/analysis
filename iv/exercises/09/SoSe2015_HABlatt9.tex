%\documentclass[12pt,a4paper]{article}
%\usepackage{a4kopka}
\documentclass[12pt,a4paper]{article}
\usepackage{ngerman}
%\usepackage{fleqn}
\usepackage{array}
\usepackage[latin1]{inputenc}
\usepackage{amssymb}
\usepackage{amsmath}
\usepackage{amsfonts}
%\usepackage{mathbbol}
\usepackage{amstext}
%\usepackage{fancyhdr}
\usepackage{relsize}
\usepackage{epic}
\usepackage{graphics}
\usepackage{graphicx}
\usepackage[T1]{fontenc}
\usepackage{a4wide}
%\usepackage{eufrak}
\usepackage{makeidx}%\usepackage{german}
\usepackage{textcomp}

\newcommand{\D}{\displaystyle}
%%%%%%%%%%%%%%%%%%%%

\newcommand{\IR}{{\mathbb{R}}}
\newcommand{\IN}{{\mathbb{N}}}
\newcommand{\IZ}{{\mathbb{Z}}}
\newcommand{\IF}{{\mathbb{F}}}
\newcommand{\IK}{{\mathbb{K}}}
\newcommand{\IQ}{{\mathbb{Q}}}
\newcommand{\IC}{{\mathbb{C}}}
\newcommand{\IP}{{\mathbb{P}}}
\newcommand{\IE}{{\mathbb{E}}}

\newcommand{\ep}{{\varepsilon}}
\newcommand{\ph}{{\varphi}}
\newcommand{\thet}{{\vartheta}}
\newcommand{\rh}{{\varrho}}
\newcommand{\de}{{\delta}}
\newcommand{\la}{{\lambda}}
\newcommand{\Om}{{\Omega}}
\newcommand{\al}{{\alpha}}
\newcommand{\be}{{\beta}}
\newcommand{\ga}{{\gamma}}
\newcommand{\om}{{\omega}}
\newcommand{\La}{{\Lambda}}
\newcommand{\Ga}{{\Gamma}}
\newcommand{\De}{{\Delta}}

\newcommand{\foh}{{\mathfrak{h}}}

\newcommand{\nach}{{\rightarrow}}
\newcommand{\Nach}{{\,\rightarrow\,}}
\newcommand{\Fou}{{\mathcal{F}}}
\newcommand{\sk}{{\,|\,}}


%\newcommand{\cosh}{\mbox{cosh}}
\newtheorem{theorem}{Theorem}[section]
\newtheorem{lemma}[theorem]{Lemma}
\newtheorem{proposition}[theorem]{Proposition}
\newtheorem{corollary}[theorem]{Corollary}
\newtheorem{remarks}[theorem]{Remarks}
%%%%%%%%%%%%%%%%%%%%

\addtolength{\voffset}{-15pt} \addtolength{\textheight}{-10pt}
%\addtolength{\textheight}{-11pt} \addtolength{\headsep}{6pt}

\setlength{\parskip}{10pt plus 2pt minus 1pt}
\setlength{\parindent}{0pt}
\newcommand{\wk}{\mbox{$\,<$\hspace{-5pt}\footnotesize )$\,$}}
\thispagestyle{empty} \addtolength{\voffset}{15pt}



\begin{document}

\rule{\textwidth}{0.3pt}
\begin{center}
\textbf{\large H\"ohere Analysis I}
\end{center}
\textbf{Sommersemester 2015 \hfill Prof. Dr. D. Lenz}

\rule{\textwidth}{0.3pt}

%%%%%%%%%%%%%%%%%%%%%%%%%%%%%%%%%%%%%%%%%%%%%%%%%%%%
%
% Hier geht's los
%
%%%%%%%%%%%%%%%%%%%%%%%%%%%%%%%%%%%%%%%%%%%%%%%%%%%%%
\textbf{Blatt 9}\hfill % Nr des Blatts
\textbf{Abgabe Dienstag 30.06.2015}

%%%%%%%%%%%%%%%%%%%%%%%%%%%%%%%%%%%%%%%%%%%%%%%%%%%%
\begin{itemize}
\item[(1)]  Sei $V$ ein Vektorraum \"uber $\mathbb{K}$ und seien $a, b : V\times V\longrightarrow \mathbb{K}$ symmetrische Sesquilinearformen mit  $|a (u,u) | \leq  b(u,u)$ f\"ur alle $u\in V$. Zeigen Sie
$$  |a(u,v)|\leq b(u,u)^{1/2} b (v,v)^{1/2}$$
f\"ur alle $u,v\in V$.\\[0.2cm]
\underline{Hinweis:} Betrachten Sie zunaechst den Fall $\mathbb{K} = \mathbb{R}$ und nutzen Sie $$a(u,v) = \frac{1}{4} ( a(u+v, u+ v) - a(u-v, u-v)).$$


\item[(2)] Seien $(X,A,\mu)$ und $(Y,B,\nu)$ Massr\"aume, $(e_j)$ eine Orthonormalbasis von $L^2(Y,\nu)$ und  $K: L^2 (Y,\nu)\longrightarrow L^2 (X,\mu)$ ein linearer  Operator mit  $\sum_j \|K e_j\|^2 <\infty$. Zeigen Sie, dass es ein messbares $k : X\times Y\longrightarrow \mathbb{C}$  mit
$$\int_{X\times Y} |k(x,y)|^2 d\mu d\nu < \infty$$
gibt mit
$$K f  = \int k(\cdot ,y) f(y) dnu (y)$$
f\"ur alle $f\in L^2 (Y,\nu)$.

\underline{Hinweis:}  Sei $(f_k)$ eine Orthonormalbasis von  $L^2 (X,\mu)$. Zeigen Sie
$\sum_{j,k} |\langle f_k, K e_j\rangle|^2 < \infty$
und definieren Sie $k:= \sum \langle f_k , K e_j \rangle  f_k  e_j$ und zeigen Sie $k\in L^2(X\times Y, \mu\times\nu)$.

\item[(3)] Sei $(H, \langle \cdot,\cdot\rangle)$ ein komplexer Hilbertraum und $s: H\times H \to \IC$ eine stetige Sesquilinearform, d.h. es existiert ein $M>0$ so dass f\"ur alle $x,y \in H$ gilt
\[ |s(x,y)| \leq M \|x\| \|y\|.\]
Zeigen Sie, dass dann ein stetiger Operator $T$ existiert, so dass f\"ur alle $x,y \in H$ gilt
\[ s(x,y) = \langle Tx, y \rangle.\]
\underline{Hinweis:} Darstellungssatz von Riesz.

\item[(4)] F�r einen separablen Hilbertraum $H$ sei ein selbstadjungierter, linearer, kompakter Operator $A:H\to H$ gegeben. Zeigen Sie, dass eine Folge von endlichen Projektionen $P_n:H\to H,\; n\in\mathbb{N}$ existiert mit $\lim_{n\to\infty}AP_n=A$ in der Operatornorm.

\end{itemize}




\textbf{Zusatzaufgabe.}\\
F�r einen Hilbertraum $H$ sei ein linearer Operator $A:H\to H$ gegeben. Zeigen Sie, dass $A$ genau dann kompakt ist, wenn eine Folge von endlichen Projektionen $P_n:H\to H,\; n\in\mathbb{N}$ existiert mit $\lim_{n\to\infty}AP_n=A$ in der Operatornorm.


\end{document}

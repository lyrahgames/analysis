%\documentclass[12pt,a4paper]{article}
%\usepackage{a4kopka}
\documentclass[12pt,a4paper]{article}
\usepackage[german]{babel}
%\usepackage{fleqn}
\usepackage{array}
\usepackage[latin1]{inputenc}
\usepackage{amssymb}
\usepackage{amsmath}
\usepackage{amsfonts}
%\usepackage{mathbbol}
\usepackage{amstext}
%\usepackage{fancyhdr}
\usepackage{relsize}
\usepackage{epic}
\usepackage{graphics}
\usepackage{graphicx}
\usepackage[T1]{fontenc}
\usepackage{a4wide}
%\usepackage{eufrak}
\usepackage{makeidx}%\usepackage{german}
\usepackage{textcomp}

\newcommand{\D}{\displaystyle}
%%%%%%%%%%%%%%%%%%%%

\newcommand{\IR}{{\mathbb{R}}}
\newcommand{\IN}{{\mathbb{N}}}
\newcommand{\IZ}{{\mathbb{Z}}}
\newcommand{\IF}{{\mathbb{F}}}
\newcommand{\IK}{{\mathbb{K}}}
\newcommand{\IQ}{{\mathbb{Q}}}
\newcommand{\IC}{{\mathbb{C}}}
\newcommand{\IP}{{\mathbb{P}}}
\newcommand{\IE}{{\mathbb{E}}}

\newcommand{\ep}{{\varepsilon}}
\newcommand{\ph}{{\varphi}}
\newcommand{\thet}{{\vartheta}}
\newcommand{\rh}{{\varrho}}
\newcommand{\de}{{\delta}}
\newcommand{\la}{{\lambda}}
\newcommand{\Om}{{\Omega}}
\newcommand{\al}{{\alpha}}
\newcommand{\be}{{\beta}}
\newcommand{\ga}{{\gamma}}
\newcommand{\om}{{\omega}}
\newcommand{\La}{{\Lambda}}
\newcommand{\Ga}{{\Gamma}}
\newcommand{\De}{{\Delta}}

\newcommand{\foh}{{\mathfrak{h}}}

\newcommand{\nach}{{\rightarrow}}
\newcommand{\Nach}{{\,\rightarrow\,}}
\newcommand{\Fou}{{\mathcal{F}}}
\newcommand{\sk}{{\,|\,}}


%\newcommand{\cosh}{\mbox{cosh}}
\newtheorem{theorem}{Theorem}[section]
\newtheorem{lemma}[theorem]{Lemma}
\newtheorem{proposition}[theorem]{Proposition}
\newtheorem{corollary}[theorem]{Corollary}
\newtheorem{remarks}[theorem]{Remarks}
%%%%%%%%%%%%%%%%%%%%

\addtolength{\voffset}{-15pt} \addtolength{\textheight}{-10pt}
%\addtolength{\textheight}{-11pt} \addtolength{\headsep}{6pt}

\setlength{\parskip}{10pt plus 2pt minus 1pt}
\setlength{\parindent}{0pt}
\newcommand{\wk}{\mbox{$\,<$\hspace{-5pt}\footnotesize )$\,$}}
\thispagestyle{empty} \addtolength{\voffset}{15pt}



\begin{document}

\rule{\textwidth}{0.3pt}
\begin{center}
\textbf{\large H\"ohere Analysis I}
\end{center}
\textbf{Sommersemester 2015 \hfill Prof. Dr. D. Lenz}

\rule{\textwidth}{0.3pt}

%%%%%%%%%%%%%%%%%%%%%%%%%%%%%%%%%%%%%%%%%%%%%%%%%%%%
%
% Hier geht's los
%
%%%%%%%%%%%%%%%%%%%%%%%%%%%%%%%%%%%%%%%%%%%%%%%%%%%%%
\textbf{Blatt 1}\hfill % Nr des Blatts
\textbf{Abgabe Dienstag 28.04.2015}

%%%%%%%%%%%%%%%%%%%%%%%%%%%%%%%%%%%%%%%%%%%%%%%%%%%%
\begin{itemize}

\item[(1)] Sei $X$ eine  unendliche Menge und $\mathcal{A}$ die
Potenzmenge von $X$. Sei $\mu : \mathcal{A}\longrightarrow
[0,\infty]$ definiert durch $\mu (A) = 0$ falls $A$ endlich   ist
und $\mu (A) = \infty$ sonst. Sei $\nu : \mathcal{A}\longrightarrow
[0,\infty]$ definiert durch $\nu (A) = 0$ falls $A$ abz\"ahlbar ist
und $\nu (A) = \infty$ sonst. Zeigen Sie, da\3 $\mu$ und $\nu$
additiv sind.  Untersuchen Sie, ob $\mu$ oder  $\nu$  Ma\3e sind. 


\item[(2)] Sei $(X,\mathcal{A})$ ein me\3barer Raum und $f :
\mathcal{A}\longrightarrow \IR$ beschr\"ankt.   Zeigen Sie die
\"Aquivalenz der folgenden beiden Aussagen:
\begin{itemize}
\item[(i)] Es ist $f$ me\3bar.

\item[(ii)] Es l\"a\3t sich $f$ gleichm\"a\3ig durch einfache
Funktionen approximieren.
\end{itemize}

\item[(3)] Seien $X$ und $Y$ Mengen,  $\mathcal{T}$ eine Topologie auf $X$ und
 $f : X\longrightarrow Y$ eine Funktion. Untersuchen Sie, ob
$$\{ B\subseteq Y : f^{-1} (B) \in \mathcal{T}\}$$
eine Topologie ist.


\item[(4)] Sei $(X,\mathcal{A})$ ein me\3barer Raum und $\mu :
\mathcal{A}\longrightarrow [0,\infty]$ erf\"ulle
\begin{itemize}
\item $\mu (\emptyset) = 0$,
\item $\mu (A\cup B) = \mu (A) + \mu (B)$ f\"ur alle $A,B\in
\mathcal{A}$ mit $A\cap B = \emptyset$.
\end{itemize}
Zeigen Sie die \"Aquivalenz der folgenden beiden Aussagen:
\begin{itemize}
\item[(i)] Es ist $\mu$ ein Ma\3.

\item[(ii)] Es gilt $\lim_{n\to \infty} \mu (A_n) = \mu\left(\bigcup_{n\in\mathbb{N}}
A_n \right)$ f\"ur alle $A_n\in \mathcal{A}$ mit $A_n \subseteq
A_{n+1}$ f\"ur alle $n\in \IN$.


\end{itemize}



\end{itemize}


\end{document}

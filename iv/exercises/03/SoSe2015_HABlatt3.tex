%\documentclass[12pt,a4paper]{article}
%\usepackage{a4kopka}
\documentclass[12pt,a4paper]{article}
\usepackage{ngerman}
%\usepackage{fleqn}
\usepackage{array}
\usepackage[latin1]{inputenc}
\usepackage{amssymb}
\usepackage{amsmath}
\usepackage{amsfonts}
%\usepackage{mathbbol}
\usepackage{amstext}
%\usepackage{fancyhdr}
\usepackage{relsize}
\usepackage{epic}
\usepackage{graphics}
\usepackage{graphicx}
\usepackage[T1]{fontenc}
\usepackage{a4wide}
%\usepackage{eufrak}
\usepackage{makeidx}%\usepackage{german}
\usepackage{textcomp}

\newcommand{\D}{\displaystyle}
%%%%%%%%%%%%%%%%%%%%

\newcommand{\IR}{{\mathbb{R}}}
\newcommand{\IN}{{\mathbb{N}}}
\newcommand{\IZ}{{\mathbb{Z}}}
\newcommand{\IF}{{\mathbb{F}}}
\newcommand{\IK}{{\mathbb{K}}}
\newcommand{\IQ}{{\mathbb{Q}}}
\newcommand{\IC}{{\mathbb{C}}}
\newcommand{\IP}{{\mathbb{P}}}
\newcommand{\IE}{{\mathbb{E}}}

\newcommand{\ep}{{\varepsilon}}
\newcommand{\ph}{{\varphi}}
\newcommand{\thet}{{\vartheta}}
\newcommand{\rh}{{\varrho}}
\newcommand{\de}{{\delta}}
\newcommand{\la}{{\lambda}}
\newcommand{\Om}{{\Omega}}
\newcommand{\al}{{\alpha}}
\newcommand{\be}{{\beta}}
\newcommand{\ga}{{\gamma}}
\newcommand{\om}{{\omega}}
\newcommand{\La}{{\Lambda}}
\newcommand{\Ga}{{\Gamma}}
\newcommand{\De}{{\Delta}}

\newcommand{\foh}{{\mathfrak{h}}}

\newcommand{\nach}{{\rightarrow}}
\newcommand{\Nach}{{\,\rightarrow\,}}
\newcommand{\Fou}{{\mathcal{F}}}
\newcommand{\sk}{{\,|\,}}


%\newcommand{\cosh}{\mbox{cosh}}
\newtheorem{theorem}{Theorem}[section]
\newtheorem{lemma}[theorem]{Lemma}
\newtheorem{proposition}[theorem]{Proposition}
\newtheorem{corollary}[theorem]{Corollary}
\newtheorem{remarks}[theorem]{Remarks}
%%%%%%%%%%%%%%%%%%%%

\addtolength{\voffset}{-15pt} \addtolength{\textheight}{-10pt}
%\addtolength{\textheight}{-11pt} \addtolength{\headsep}{6pt}

\setlength{\parskip}{10pt plus 2pt minus 1pt}
\setlength{\parindent}{0pt}
\newcommand{\wk}{\mbox{$\,<$\hspace{-5pt}\footnotesize )$\,$}}
\thispagestyle{empty} \addtolength{\voffset}{15pt}



\begin{document}

\rule{\textwidth}{0.3pt}
\begin{center}
\textbf{\large H\"ohere Analysis I}
\end{center}
\textbf{Sommersemester 2015 \hfill Prof. Dr. D. Lenz}

\rule{\textwidth}{0.3pt}

%%%%%%%%%%%%%%%%%%%%%%%%%%%%%%%%%%%%%%%%%%%%%%%%%%%%
%
% Hier geht's los
%
%%%%%%%%%%%%%%%%%%%%%%%%%%%%%%%%%%%%%%%%%%%%%%%%%%%%%
\textbf{Blatt 3}\hfill % Nr des Blatts
\textbf{Abgabe Dienstag 12.05.2015}

%%%%%%%%%%%%%%%%%%%%%%%%%%%%%%%%%%%%%%%%%%%%%%%%%%%%
\begin{itemize}
\item[(1)] Sei $(X,\mathcal{A},\mu)$ ein Ma\ss raum. 
\begin{itemize}
\item[(a)] Zeigen Sie f�r eine beliebige Folge $A_n\in\mathcal{A},\; n\in\mathbb{N}$, dass die Ungleichung $\mu\left( \cup_{n\in\mathbb{N}} A_n\right)\leq \sum_{n\in\mathbb{N}} \mu(A_n)$ gilt.
\item[(b)] Zeigen Sie, dass f\"ur eine messbare Funktion $f:X\to\mathbb{C}$ die folgenden Aussagen \"aquivalent sind.
\begin{itemize}
\item[(i)] Die Funktion $f:X\to\mathbb{C}$ verschwindet $\mu$-fast \"uberall.
\item[(ii)] F\"ur ein $p\geq 1$ gilt $\int_X|f|^p\, d\mu=0$.
\item[(ii)] F\"ur alle $p\geq 1$ gilt $\int_X|f|^p\, d\mu=0$.
\end{itemize}
\end{itemize} 
\item[(2)] Sei $(X,\mathcal{A},\mu)$ ein Ma\ss raum. Gegeben seien $f,g\in\mathcal{L}^1(X,\mathcal{A},\mu)$ und $\alpha,\beta\in\mathbb{C}$. Zeigen Sie, dass die Gleichheit
	$$
	\int\limits_X (\alpha f+\beta g)\; d\mu\; =\; \alpha \int\limits_X f \; d\mu+\beta \int\limits_Xg\; d\mu
	$$
	gilt.

\item[(3)] F\"ur $0<p\leq\infty$ definieren wir die Abbildung $\|\cdot\|_p:\mathbb{R}^2\to[0,\infty)$ durch
$$
\|(x,y)\|_p:=\begin{cases}
\left(|x|^p+|y|^p\right)^{\frac{1}{p}}, \qquad &\text{ falls }\; 0<p<\infty,\\
\max\{|x|,|y|\},\qquad&\text{ falls }\; p=\infty,
\end{cases},\qquad\quad (x,y)\in\mathbb{R}^2.
$$
\begin{itemize}
\item[(i)] Zeichnen Sie die Mengen
$$
B_1^{(p)}(0):=\{(x,y)\in\mathbb{R}^2\;:\; \|(x,y)\|_p\leq 1\}
$$
f\"ur $p=\frac{1}{2},1,2,\infty$.
\item[(ii)] Zeigen Sie, dass f\"ur beliebige $(x,y)\in\mathbb{R}^2$ die Gleichheit
$$
\|(x,y)\|_\infty=\lim\limits_{p\to\infty}\|(x,y)\|_p
$$
gilt.
\item[(iii)] Zeigen Sie, dass f\"ur $0<p<1$ die Abbildung $\|\cdot\|_p$ keine Norm ist.
\end{itemize}

\item[(4)] Sei $(X,\mathcal{A},\mu)$ ein Ma\ss raum. Zeigen Sie, dass der Raum $L^\infty(X,\mathcal{A},\mu)$ vollst\"andig ist.
\end{itemize}

\textbf{Zusatz}\\
Gegeben sei ein me\ss barer Raum $(X,\mathcal{A},\mu)$. Zeigen Sie, dass f\"ur eine me\ss bare Funktion $f:X\to\mathbb{C}$ mit $f\in L^p(X,\mathcal{A},\mu)$ f\"ur alle $1\leq p\leq \infty$ die Gleichheit
$$
\|f\|_\infty=\lim\limits_{p\to\infty}\|f\|_p
$$
gilt.
\end{document}

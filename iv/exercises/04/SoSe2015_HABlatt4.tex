%\documentclass[12pt,a4paper]{article}
%\usepackage{a4kopka}
\documentclass[12pt,a4paper]{article}
\usepackage{ngerman}
%\usepackage{fleqn}
\usepackage{array}
\usepackage[latin1]{inputenc}
\usepackage{amssymb}
\usepackage{amsmath}
\usepackage{amsfonts}
%\usepackage{mathbbol}
\usepackage{amstext}
%\usepackage{fancyhdr}
\usepackage{relsize}
\usepackage{epic}
\usepackage{graphics}
\usepackage{graphicx}
\usepackage[T1]{fontenc}
\usepackage{a4wide}
%\usepackage{eufrak}
\usepackage{makeidx}%\usepackage{german}
\usepackage{textcomp}

\newcommand{\D}{\displaystyle}
%%%%%%%%%%%%%%%%%%%%

\newcommand{\IR}{{\mathbb{R}}}
\newcommand{\IN}{{\mathbb{N}}}
\newcommand{\IZ}{{\mathbb{Z}}}
\newcommand{\IF}{{\mathbb{F}}}
\newcommand{\IK}{{\mathbb{K}}}
\newcommand{\IQ}{{\mathbb{Q}}}
\newcommand{\IC}{{\mathbb{C}}}
\newcommand{\IP}{{\mathbb{P}}}
\newcommand{\IE}{{\mathbb{E}}}

\newcommand{\ep}{{\varepsilon}}
\newcommand{\ph}{{\varphi}}
\newcommand{\thet}{{\vartheta}}
\newcommand{\rh}{{\varrho}}
\newcommand{\de}{{\delta}}
\newcommand{\la}{{\lambda}}
\newcommand{\Om}{{\Omega}}
\newcommand{\al}{{\alpha}}
\newcommand{\be}{{\beta}}
\newcommand{\ga}{{\gamma}}
\newcommand{\om}{{\omega}}
\newcommand{\La}{{\Lambda}}
\newcommand{\Ga}{{\Gamma}}
\newcommand{\De}{{\Delta}}

\newcommand{\foh}{{\mathfrak{h}}}

\newcommand{\nach}{{\rightarrow}}
\newcommand{\Nach}{{\,\rightarrow\,}}
\newcommand{\Fou}{{\mathcal{F}}}
\newcommand{\sk}{{\,|\,}}


%\newcommand{\cosh}{\mbox{cosh}}
\newtheorem{theorem}{Theorem}[section]
\newtheorem{lemma}[theorem]{Lemma}
\newtheorem{proposition}[theorem]{Proposition}
\newtheorem{corollary}[theorem]{Corollary}
\newtheorem{remarks}[theorem]{Remarks}
%%%%%%%%%%%%%%%%%%%%

\addtolength{\voffset}{-15pt} \addtolength{\textheight}{-10pt}
%\addtolength{\textheight}{-11pt} \addtolength{\headsep}{6pt}

\setlength{\parskip}{10pt plus 2pt minus 1pt}
\setlength{\parindent}{0pt}
\newcommand{\wk}{\mbox{$\,<$\hspace{-5pt}\footnotesize )$\,$}}
\thispagestyle{empty} \addtolength{\voffset}{15pt}



\begin{document}

\rule{\textwidth}{0.3pt}
\begin{center}
\textbf{\large H\"ohere Analysis I}
\end{center}
\textbf{Sommersemester 2015 \hfill Prof. Dr. D. Lenz}

\rule{\textwidth}{0.3pt}

%%%%%%%%%%%%%%%%%%%%%%%%%%%%%%%%%%%%%%%%%%%%%%%%%%%%
%
% Hier geht's los
%
%%%%%%%%%%%%%%%%%%%%%%%%%%%%%%%%%%%%%%%%%%%%%%%%%%%%%
\textbf{Blatt 4}\hfill % Nr des Blatts
\textbf{Abgabe Dienstag 19.05.2015}

%%%%%%%%%%%%%%%%%%%%%%%%%%%%%%%%%%%%%%%%%%%%%%%%%%%%
\begin{itemize}
\item[(1)] Sei $\mathbb{R}$ ausgestattet mit der Borel $\sigma$-algebra und dem Lebesguema\ss $\,\lambda$. Zeigen Sie, dass f�r $p,q\in\mathbb{N}$ beliebig mit $p\neq q$ nicht die Inklusion $L^p(\mathbb{R},\mathcal{B}(\mathbb{R}),\lambda)\subseteq L^q(\mathbb{R},\mathcal{B}(\mathbb{R}),\lambda)$ gilt.
\item[(2)] Gegeben sei der Raum $\mathcal{C}([0,1])$ der stetigen Funktionen auf den Intervall $[0,1]$.
    \begin{itemize}
    \item[(a)] Zeigen Sie, dass 
    $$
    \langle f,g\rangle:=\int\limits_0^1 f(x) \overline{g(x)}\; dx,\qquad f,g\in \mathcal{C}([0,1]),
    $$
    ein Skalarprodukt auf $\mathcal{C}([0,1])$ ist.
    \item[(b)] Beweisen Sie, dass $\mathcal{C}([0,1])$ mit dem eben definierten Skalarprodukt kein Hilbertraum ist.
    \end{itemize}

\item[(3)] Zeigen Sie:

\begin{itemize}
\item[(a)] Die Normen $\|\cdot\|_p$ auf $\ell^p$ werden f\"ur $p\neq 2$ nicht von einem Skalarprodukt induziert.
\item[(b)] Die Supremumsnorm auf $C([0,1])$ wird nicht durch ein Skalarprodukt induziert.
\end{itemize}
\underline{Hinweis:} In Hilbertr�umen gilt die Parallelogrammidentit�t.

\item[(4)] Sei $V$ ein Vektorraum und $s$ ein Skalarprodukt auf $V$. Zeigen Sie, dass
%
$$ |s(x,y)| = s(x,x)^{\frac{1}{2}}s(y,y)^{\frac{1}{2}}$$
%
genau dann gilt, wenn $x$ und $y$ linear abh\"angig sind.

\end{itemize}


\end{document}

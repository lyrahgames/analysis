%\documentclass[12pt,a4paper]{article}
%\usepackage{a4kopka}
\documentclass[12pt,a4paper]{article}
\usepackage{ngerman}
%\usepackage{fleqn}
\usepackage{array}
\usepackage[latin1]{inputenc}
\usepackage{amssymb}
\usepackage{amsmath}
\usepackage{amsfonts}
%\usepackage{mathbbol}
\usepackage{amstext}
%\usepackage{fancyhdr}
\usepackage{relsize}
\usepackage{epic}
\usepackage{graphics}
\usepackage{graphicx}
\usepackage[T1]{fontenc}
\usepackage{a4wide}
%\usepackage{eufrak}
\usepackage{makeidx}%\usepackage{german}
\usepackage{textcomp}

\newcommand{\D}{\displaystyle}
%%%%%%%%%%%%%%%%%%%%

\newcommand{\IR}{{\mathbb{R}}}
\newcommand{\IN}{{\mathbb{N}}}
\newcommand{\IZ}{{\mathbb{Z}}}
\newcommand{\IF}{{\mathbb{F}}}
\newcommand{\IK}{{\mathbb{K}}}
\newcommand{\IQ}{{\mathbb{Q}}}
\newcommand{\IC}{{\mathbb{C}}}
\newcommand{\IP}{{\mathbb{P}}}
\newcommand{\IE}{{\mathbb{E}}}

\newcommand{\ep}{{\varepsilon}}
\newcommand{\ph}{{\varphi}}
\newcommand{\thet}{{\vartheta}}
\newcommand{\rh}{{\varrho}}
\newcommand{\de}{{\delta}}
\newcommand{\la}{{\lambda}}
\newcommand{\Om}{{\Omega}}
\newcommand{\al}{{\alpha}}
\newcommand{\be}{{\beta}}
\newcommand{\ga}{{\gamma}}
\newcommand{\om}{{\omega}}
\newcommand{\La}{{\Lambda}}
\newcommand{\Ga}{{\Gamma}}
\newcommand{\De}{{\Delta}}

\newcommand{\foh}{{\mathfrak{h}}}

\newcommand{\nach}{{\rightarrow}}
\newcommand{\Nach}{{\,\rightarrow\,}}
\newcommand{\Fou}{{\mathcal{F}}}
\newcommand{\sk}{{\,|\,}}


%\newcommand{\cosh}{\mbox{cosh}}
\newtheorem{theorem}{Theorem}[section]
\newtheorem{lemma}[theorem]{Lemma}
\newtheorem{proposition}[theorem]{Proposition}
\newtheorem{corollary}[theorem]{Corollary}
\newtheorem{remarks}[theorem]{Remarks}
%%%%%%%%%%%%%%%%%%%%

\addtolength{\voffset}{-15pt} \addtolength{\textheight}{-10pt}
%\addtolength{\textheight}{-11pt} \addtolength{\headsep}{6pt}

\setlength{\parskip}{10pt plus 2pt minus 1pt}
\setlength{\parindent}{0pt}
\newcommand{\wk}{\mbox{$\,<$\hspace{-5pt}\footnotesize )$\,$}}
\thispagestyle{empty} \addtolength{\voffset}{15pt}



\begin{document}

\rule{\textwidth}{0.3pt}
\begin{center}
\textbf{\large H\"ohere Analysis I}
\end{center}
\textbf{Sommersemester 2015 \hfill Prof. Dr. D. Lenz}

\rule{\textwidth}{0.3pt}

%%%%%%%%%%%%%%%%%%%%%%%%%%%%%%%%%%%%%%%%%%%%%%%%%%%%
%
% Hier geht's los
%
%%%%%%%%%%%%%%%%%%%%%%%%%%%%%%%%%%%%%%%%%%%%%%%%%%%%%
\textbf{Blatt 7}\hfill % Nr des Blatts
\textbf{Abgabe Dienstag 09.06.2015}

%%%%%%%%%%%%%%%%%%%%%%%%%%%%%%%%%%%%%%%%%%%%%%%%%%%%
\begin{itemize}
\item[(1)] Zeigen Sie, dass jeder endlichdimensionale Teilraum eines Hilbertraumes abgeschlossen ist.

\item[(2)] Zeigen Sie, dass das orthogonale Komplement einer beliebigen Menge in einem  Vektorraum mit Skalarprodukt ein abgeschlossener Unterraum ist.

\item[(3)] Es sei $H$ ein Hilbertraum und $U,V\subseteq H$ abgeschlossene Unterr\"aume. Weiterhin seien $P_U,P_V$ die zugeh\"origen Orthogonalprojektionen. Zeigen Sie folgende Aussagen.
\begin{itemize}
\item[(a)] Es gilt $P_UP_V=0$ genau dann, wenn $U\perp V$.
\item[(b)] Es ist $P_U+P_V$ eine Orthogonalprojektion genau dann, wenn $P_UP_V=0$.
\item[(c)] Es ist $P_UP_V$ eine Orthogonalprojektion genau dann, wenn $P_UP_V=P_VP_U$.
\item[(d)] Es gilt $P_UP_V=P_V$ genau dann, wenn $V\subseteq U$.
\item[(e)] Es ist $P_U-P_V$ eine Orthogonalprojektion genau dann, wenn $V\subseteq U$.
\item[(f)] Es gilt $P_UP_V=P_V$ genau dann, wenn f\"ur alle $x\in H$ die Ungleichung $\|P_Vx\|\leq\|P_Ux\|$ gilt. 
\end{itemize}
\item[(4)] Es sei $(e_n)_{n\in\mathbb{N}}$ ein Orthonormalsystem in dem Hilbertraum $H$ mit Skalarprodukt $\langle\cdot,\cdot\rangle:H \times H\to \mathbb{C}$. 
\begin{itemize}
\item[(a)] Zeigen Sie, dass $(e_n)$ schwach gegen Null konvergiert (also f\"ur alle $x\in H$ gilt $\lim\limits_{n\to\infty}\langle x,e_n\rangle=0$).
\item[(b)] Zeigen Sie, dass $(e_n)$ nicht in Norm gegen $0$ konvergiert.
\end{itemize}

\textbf{Zusatz}\\
Gegeben sei ein Hilbertraum $H$ mit Skalarprodukt $\langle\cdot,\cdot\rangle:H \times H\to \mathbb{C}$. Zeigen Sie die folgenden Aussagen.
\begin{itemize}
\item[(a)] Ist $(e_\alpha)_{\alpha\in I}$ ein Orthonormalsystem in $H$ f�r eine gegebenene Indexmenge $I$, dann gilt 
$$
\|e_\alpha-e_\beta\|=2,\qquad \alpha\neq\beta.
$$
\item[(b)] Gibt es eine abz\"ahlbare Orthonormalbasis von $H$, so gibt es eine abz\"ahlbare dichte Teilmenge von $H$.
\item[(c)] Falls $H$ eine Orthonormalbasis $(e_\alpha)_{\alpha\in I}$ mit \"uberabz\"ahlbarer Indexmenge besitzt, dann gibt es keine abz\"ahlbare Orthonormalbasis f\"ur $H$.
\end{itemize}
\end{itemize}


\end{document}

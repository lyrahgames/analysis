%\documentclass[12pt,a4paper]{article}
%\usepackage{a4kopka}
\documentclass[12pt,a4paper]{article}
\usepackage{ngerman}
%\usepackage{fleqn}
\usepackage{array}
\usepackage[latin1]{inputenc}
\usepackage{amssymb}
\usepackage{amsmath}
\usepackage{amsfonts}
%\usepackage{mathbbol}
\usepackage{amstext}
%\usepackage{fancyhdr}
\usepackage{relsize}
\usepackage{epic}
\usepackage{graphics}
\usepackage{graphicx}
\usepackage[T1]{fontenc}
\usepackage{a4wide}
%\usepackage{eufrak}
\usepackage{makeidx}%\usepackage{german}
\usepackage{textcomp}

\newcommand{\D}{\displaystyle}
%%%%%%%%%%%%%%%%%%%%

\newcommand{\IR}{{\mathbb{R}}}
\newcommand{\IN}{{\mathbb{N}}}
\newcommand{\IZ}{{\mathbb{Z}}}
\newcommand{\IF}{{\mathbb{F}}}
\newcommand{\IK}{{\mathbb{K}}}
\newcommand{\IQ}{{\mathbb{Q}}}
\newcommand{\IC}{{\mathbb{C}}}
\newcommand{\IP}{{\mathbb{P}}}
\newcommand{\IE}{{\mathbb{E}}}

\newcommand{\ep}{{\varepsilon}}
\newcommand{\ph}{{\varphi}}
\newcommand{\thet}{{\vartheta}}
\newcommand{\rh}{{\varrho}}
\newcommand{\de}{{\delta}}
\newcommand{\la}{{\lambda}}
\newcommand{\Om}{{\Omega}}
\newcommand{\al}{{\alpha}}
\newcommand{\be}{{\beta}}
\newcommand{\ga}{{\gamma}}
\newcommand{\om}{{\omega}}
\newcommand{\La}{{\Lambda}}
\newcommand{\Ga}{{\Gamma}}
\newcommand{\De}{{\Delta}}

\newcommand{\foh}{{\mathfrak{h}}}

\newcommand{\nach}{{\rightarrow}}
\newcommand{\Nach}{{\,\rightarrow\,}}
\newcommand{\Fou}{{\mathcal{F}}}
\newcommand{\sk}{{\,|\,}}


%\newcommand{\cosh}{\mbox{cosh}}
\newtheorem{theorem}{Theorem}[section]
\newtheorem{lemma}[theorem]{Lemma}
\newtheorem{proposition}[theorem]{Proposition}
\newtheorem{corollary}[theorem]{Corollary}
\newtheorem{remarks}[theorem]{Remarks}
%%%%%%%%%%%%%%%%%%%%

\addtolength{\voffset}{-15pt} \addtolength{\textheight}{-10pt}
%\addtolength{\textheight}{-11pt} \addtolength{\headsep}{6pt}

\setlength{\parskip}{10pt plus 2pt minus 1pt}
\setlength{\parindent}{0pt}
\newcommand{\wk}{\mbox{$\,<$\hspace{-5pt}\footnotesize )$\,$}}
\thispagestyle{empty} \addtolength{\voffset}{15pt}



\begin{document}

\rule{\textwidth}{0.3pt}
\begin{center}
\textbf{\large H\"ohere Analysis I}
\end{center}
\textbf{Sommersemester 2015 \hfill Prof. Dr. D. Lenz}

\rule{\textwidth}{0.3pt}

%%%%%%%%%%%%%%%%%%%%%%%%%%%%%%%%%%%%%%%%%%%%%%%%%%%%
%
% Hier geht's los
%
%%%%%%%%%%%%%%%%%%%%%%%%%%%%%%%%%%%%%%%%%%%%%%%%%%%%%
\textbf{Blatt 8}\hfill % Nr des Blatts
\textbf{Abgabe Dienstag 23.06.2015}

%%%%%%%%%%%%%%%%%%%%%%%%%%%%%%%%%%%%%%%%%%%%%%%%%%%%
\begin{itemize}
\item[(1)] Gegeben seien  Hilbertr\"aume $H$, $K$ und $L$.

(a) Zeigen Sie f\"ur  alle beschr\"anken linearen Operatoren $A,B$
von $H$ nach $K$  und $\lambda \in \mathbb{K}$ die Aussagen:
$$(A^\ast)^\ast=A\;\:\mbox{und}\;\: (A+\lambda B)^\ast= A^\ast +
 \overline{\lambda}B^\ast.$$

(b) Zeigen Sie f\"ur alle beschr\"ankten linearen Operatoren $A$ von
$K$ nach $L$ und $B$ von $H$ nach $K$ $$(AB)^\ast=B^\ast A^\ast.$$


\item[(2)]  Gegeben seien  Hilbertr\"aume $H$, $K$ und $L$. Zeigen
Sie:

(a) F\"ur alle  beschr\"ankten linearen Operatoren $A,B$ von $H$
nach $K$ gilt $\| A + B\|\leq \|A\| + \|B\|.$

(b) F\"ur alle beschr\"ankten linearen  Operatoren $A$ von $H$ nach
$K$ und $B$ von $K$ nach $L$ gilt $\| B A\| \leq \|B\| \|A\|$.



\item[(3)] Sei $ \varphi : \IN\longrightarrow \IR$ beschraenkt und
$$M_\varphi : \ell^2 \longrightarrow \ell^2, \;\: f\mapsto \varphi f,$$ der Operator der
Multiplikation mit $\varphi$. Bestimmen Sie das Spektrum von
$M_\varphi$.



\item[(4)] Sei $H$ ein Hilbertraum und $T$ ein linearer beschr\"ankter Operator von $H$ nach $H$.
Zeigen Sie: Gilt $\|T\|<1$, so ist $I - T$ invertierbar mit Inverser
gegeben durch  $(I-T)^{-1}=\sum_{k=0}^\infty T^k$.

\end{itemize}




\textbf{Zusatzaufgabe.}\\
In einem Hilbertraum $H$ enth\"alt jede beschr\"ankte Folge $(x_n)$
eine Teilfolge $(x_{n_k})$, sodass f�r jedes $y\in H$ die Folge
$k\mapsto \langle x_{n_k},y\rangle$ konvergiert.

\underline{Hinweis:} Es reicht (Warum?) sich auf den separablen Fall
zu beschr\"anken.




\end{document}

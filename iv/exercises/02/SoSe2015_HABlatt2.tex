%\documentclass[12pt,a4paper]{article}
%\usepackage{a4kopka}
\documentclass[12pt,a4paper]{article}
\usepackage{ngerman}
%\usepackage{fleqn}
\usepackage{array}
\usepackage[latin1]{inputenc}
\usepackage{amssymb}
\usepackage{amsmath}
\usepackage{amsfonts}
%\usepackage{mathbbol}
\usepackage{amstext}
%\usepackage{fancyhdr}
\usepackage{relsize}
\usepackage{epic}
\usepackage{graphics}
\usepackage{graphicx}
\usepackage[T1]{fontenc}
\usepackage{a4wide}
%\usepackage{eufrak}
\usepackage{makeidx}%\usepackage{german}
\usepackage{textcomp}

\newcommand{\D}{\displaystyle}
%%%%%%%%%%%%%%%%%%%%

\newcommand{\IR}{{\mathbb{R}}}
\newcommand{\IN}{{\mathbb{N}}}
\newcommand{\IZ}{{\mathbb{Z}}}
\newcommand{\IF}{{\mathbb{F}}}
\newcommand{\IK}{{\mathbb{K}}}
\newcommand{\IQ}{{\mathbb{Q}}}
\newcommand{\IC}{{\mathbb{C}}}
\newcommand{\IP}{{\mathbb{P}}}
\newcommand{\IE}{{\mathbb{E}}}

\newcommand{\ep}{{\varepsilon}}
\newcommand{\ph}{{\varphi}}
\newcommand{\thet}{{\vartheta}}
\newcommand{\rh}{{\varrho}}
\newcommand{\de}{{\delta}}
\newcommand{\la}{{\lambda}}
\newcommand{\Om}{{\Omega}}
\newcommand{\al}{{\alpha}}
\newcommand{\be}{{\beta}}
\newcommand{\ga}{{\gamma}}
\newcommand{\om}{{\omega}}
\newcommand{\La}{{\Lambda}}
\newcommand{\Ga}{{\Gamma}}
\newcommand{\De}{{\Delta}}

\newcommand{\foh}{{\mathfrak{h}}}

\newcommand{\nach}{{\rightarrow}}
\newcommand{\Nach}{{\,\rightarrow\,}}
\newcommand{\Fou}{{\mathcal{F}}}
\newcommand{\sk}{{\,|\,}}


%\newcommand{\cosh}{\mbox{cosh}}
\newtheorem{theorem}{Theorem}[section]
\newtheorem{lemma}[theorem]{Lemma}
\newtheorem{proposition}[theorem]{Proposition}
\newtheorem{corollary}[theorem]{Corollary}
\newtheorem{remarks}[theorem]{Remarks}
%%%%%%%%%%%%%%%%%%%%

\addtolength{\voffset}{-15pt} \addtolength{\textheight}{-10pt}
%\addtolength{\textheight}{-11pt} \addtolength{\headsep}{6pt}

\setlength{\parskip}{10pt plus 2pt minus 1pt}
\setlength{\parindent}{0pt}
\newcommand{\wk}{\mbox{$\,<$\hspace{-5pt}\footnotesize )$\,$}}
\thispagestyle{empty} \addtolength{\voffset}{15pt}



\begin{document}

\rule{\textwidth}{0.3pt}
\begin{center}
\textbf{\large H\"ohere Analysis I}
\end{center}
\textbf{Sommersemester 2015 \hfill Prof. Dr. D. Lenz}

\rule{\textwidth}{0.3pt}

%%%%%%%%%%%%%%%%%%%%%%%%%%%%%%%%%%%%%%%%%%%%%%%%%%%%
%
% Hier geht's los
%
%%%%%%%%%%%%%%%%%%%%%%%%%%%%%%%%%%%%%%%%%%%%%%%%%%%%%
\textbf{Blatt 2}\hfill % Nr des Blatts
\textbf{Abgabe Dienstag 05.05.2015}

%%%%%%%%%%%%%%%%%%%%%%%%%%%%%%%%%%%%%%%%%%%%%%%%%%%%
\begin{itemize}

\item[(1)] Es seien Ma\ss e $\mu_n,\; n\in\mathbb{N}$ auf einen messbaren Raum $(X,\mathcal{A})$ gegeben. Zeigen Sie, dass durch $\mu(A):=\lim\limits_{n\to\infty} \mu_n(A)$ ein Ma\ss\, definiert wird, falls $\mu_n\leq\mu_{n+1}$ f\"ur alle $n\in\mathbb{N}$ gilt.\\
\underline{Hinweis:} Es gilt $\mu_n\leq\mu_{n+1}$, falls f�r alle $A\in\mathcal{A}$ die Ungleichung $\mu_n(A)\leq\mu_{n+1}(A)$ gilt.

\item[(2)] Gegeben seien Ma\ss e $\mu_n,\; n\in\mathbb{N}$ auf einen messbaren Raum $(X,\mathcal{A})$. Zeigen Sie f�r beliebige $a_n\geq 0,\; n\in\mathbb{N}$, dass $\sum\limits_{n\in\mathbb{N}}a_n\cdot\mu_n$ ein Ma\ss\, bildet.

\item[(3)] Zeigen Sie, dass jede abz\"ahlbare Teilmenge von $\mathbb{R}$ eine Borelmenge ist. Untersuchen Sie die Funktion 
$$
f(x):=\begin{cases}
x,\qquad x\in\mathbb{Q},\\
0,\qquad x\in\mathbb{R}\setminus\mathbb{Q},
\end{cases}
$$
auf Messbarkeit.

\item[(4)] Betrachten Sie die reelen Zahlen $\mathbb{R}$ ausgestattet mit der Borel $\sigma$-algebra $\mathcal{B}(\mathbb{R})$. F\"ur eine Folge $x_n\in\mathbb{R},\; n\in\mathbb{N}$ definieren wir die Abbildung $\mu:\mathcal{B}(\mathbb{R})\to[0,\infty]$ durch $\mu:=\sum\limits_{n\in\mathbb{N}} \delta_{x_n}$ wobei 
$$\delta_{x_n}(A):= \begin{cases}
1,\qquad x_n\in A,\\
0,\qquad \text{sonst},
\end{cases}\qquad A\in\mathcal{B}(\mathbb{R}).
$$ 
\begin{itemize}
\item[(a)] Wann gilt f\"ur beschr\"ankte Intervalle $I\subsetneq\mathbb{R}$, dass $\mu(I)<\infty$?
\item[(b)] Welche Eigenschaften muss die Folge $(x_n)_{n\in\mathbb{N}}$ besitzen damit $\mu$ $\sigma$-endlich ist? 
\underline{Hinweis:} Ein Ma\ss\, $\mu$ hei\ss t $\sigma$-endlich, falls messbare Mengen $A_n\in\mathcal{B}(\mathbb{R}),\; n\in\mathbb{N}$ existieren mit $\mathbb{R}=\cup_{n\in\mathbb{N}}A_n$ und $\mu(A_n)<\infty$.
\end{itemize}
\textbf{Zusatz}\\
Die Cantormenge $C$ entsteht aus dem Intervall $[0,1]$, indem zun\"achst das offene mittlere Drittel herausgenommen wird, aus den zwei verbleibenden Intervallen wieder jeweils das offene Drittel herausgenommen wird, usw., also
$$
C:=[0,1]\setminus \left( \left(\frac{1}{3},\frac{2}{3}\right)\cup\left(\frac{1}{9},\frac{2}{9}\right) \cup \left(\frac{7}{9},\frac{8}{9}\right) \cup \ldots \right)
$$
Zeigen Sie die folgenden Aussagen.
\begin{itemize}
\item[a)] Die Menge $C$ ist eine Lebesgue Nullmenge.
\item[b)] Die Menge $C$ besteht genau aus den Punkten $a\in\IR$ mit $a=\sum\limits_{j=1}^\infty a_j 3^{-j}$ mit $a_j\in\{0,2\}$.
\item[c)] Die Menge $C$ ist \"uberabz\"ahlbar.
\end{itemize}
\end{itemize}



\end{document}

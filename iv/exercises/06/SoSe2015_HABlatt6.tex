%\documentclass[12pt,a4paper]{article}
%\usepackage{a4kopka}
\documentclass[12pt,a4paper]{article}
\usepackage{ngerman}
%\usepackage{fleqn}
\usepackage{array}
\usepackage[latin1]{inputenc}
\usepackage{amssymb}
\usepackage{amsmath}
\usepackage{amsfonts}
%\usepackage{mathbbol}
\usepackage{amstext}
%\usepackage{fancyhdr}
\usepackage{relsize}
\usepackage{epic}
\usepackage{graphics}
\usepackage{graphicx}
\usepackage[T1]{fontenc}
\usepackage{a4wide}
%\usepackage{eufrak}
\usepackage{makeidx}%\usepackage{german}
\usepackage{textcomp}

\newcommand{\D}{\displaystyle}
%%%%%%%%%%%%%%%%%%%%

\newcommand{\IR}{{\mathbb{R}}}
\newcommand{\IN}{{\mathbb{N}}}
\newcommand{\IZ}{{\mathbb{Z}}}
\newcommand{\IF}{{\mathbb{F}}}
\newcommand{\IK}{{\mathbb{K}}}
\newcommand{\IQ}{{\mathbb{Q}}}
\newcommand{\IC}{{\mathbb{C}}}
\newcommand{\IP}{{\mathbb{P}}}
\newcommand{\IE}{{\mathbb{E}}}

\newcommand{\ep}{{\varepsilon}}
\newcommand{\ph}{{\varphi}}
\newcommand{\thet}{{\vartheta}}
\newcommand{\rh}{{\varrho}}
\newcommand{\de}{{\delta}}
\newcommand{\la}{{\lambda}}
\newcommand{\Om}{{\Omega}}
\newcommand{\al}{{\alpha}}
\newcommand{\be}{{\beta}}
\newcommand{\ga}{{\gamma}}
\newcommand{\om}{{\omega}}
\newcommand{\La}{{\Lambda}}
\newcommand{\Ga}{{\Gamma}}
\newcommand{\De}{{\Delta}}

\newcommand{\foh}{{\mathfrak{h}}}

\newcommand{\nach}{{\rightarrow}}
\newcommand{\Nach}{{\,\rightarrow\,}}
\newcommand{\Fou}{{\mathcal{F}}}
\newcommand{\sk}{{\,|\,}}


%\newcommand{\cosh}{\mbox{cosh}}
\newtheorem{theorem}{Theorem}[section]
\newtheorem{lemma}[theorem]{Lemma}
\newtheorem{proposition}[theorem]{Proposition}
\newtheorem{corollary}[theorem]{Corollary}
\newtheorem{remarks}[theorem]{Remarks}
%%%%%%%%%%%%%%%%%%%%

\addtolength{\voffset}{-15pt} \addtolength{\textheight}{-10pt}
%\addtolength{\textheight}{-11pt} \addtolength{\headsep}{6pt}

\setlength{\parskip}{10pt plus 2pt minus 1pt}
\setlength{\parindent}{0pt}
\newcommand{\wk}{\mbox{$\,<$\hspace{-5pt}\footnotesize )$\,$}}
\thispagestyle{empty} \addtolength{\voffset}{15pt}



\begin{document}

\rule{\textwidth}{0.3pt}
\begin{center}
\textbf{\large H\"ohere Analysis I}
\end{center}
\textbf{Sommersemester 2015 \hfill Prof. Dr. D. Lenz}

\rule{\textwidth}{0.3pt}

%%%%%%%%%%%%%%%%%%%%%%%%%%%%%%%%%%%%%%%%%%%%%%%%%%%%
%
% Hier geht's los
%
%%%%%%%%%%%%%%%%%%%%%%%%%%%%%%%%%%%%%%%%%%%%%%%%%%%%%
\textbf{Blatt 6}\hfill % Nr des Blatts
\textbf{Abgabe Dienstag 02.06.2015}

%%%%%%%%%%%%%%%%%%%%%%%%%%%%%%%%%%%%%%%%%%%%%%%%%%%%
\begin{itemize}
\item[(1)] Finden Sie jeweils ein Beispiel eines Hilbertraumes $(H,\langle\cdot,\cdot \rangle)$ und einer Teilmenge $A\subseteq H$ mit einem $x\in H\setminus A$, so dass gilt:
    \begin{itemize}
    \item[(a)] $A$ ist konvex aber nicht abgeschlossen und es gibt keine beste Approximation von $x$ in $A$.
    \item[(b)] $A$ ist abgeschlossen aber nicht konvex und es gibt mehr als eine beste Approximation von $x$ in $A$.
    \item[(c)] $A$ ist abgeschlossen aber nicht konvex und es gibt keine beste Approximation von $x$ in $A$.
    \end{itemize}
K�nnen Sie in (c) auch ein Beispiel mit einem endlich dimensionalen Hilbertraum angeben?

\item[(2)] Zeigen Sie, dass der Approximationssatz in $\ell^\infty$ nicht gilt.\\
\underline{Hinweis:} Finden Sie eine Teilmenge von $\ell^\infty$, so dass es zu gegebenen $x\in \ell^\infty$ mehrere beste Approximationen gibt.

\item[(3)] Gegeben sei ein Vektorraum $X$ mit einem Semi-Skalarprodukt $\langle\cdot,\cdot \rangle$ und $N:=\{x\in X\;|\; \langle x,x\rangle =0\}$. Zeigen Sie, dass auf dem Quotientenraum $X/N$ durch
	$$
	\langle [x],[y]\rangle:=\langle x,y\rangle
	$$
	f�r Elemente $[x], [y]\in X/N$ ein Skalarprodukt definiert wird.

\item[(4)] Sei $(H,\langle\cdot,\cdot \rangle)$ ein Hilbertraum und $P:H\to H$ eine lineare Abbildung. Zeigen Sie folgende Aussage. Es ist $P$ genau dann die orthogonale Projektion auf einen abgeschlossenen Unterraum, wenn $P=P^2$ und  $\langle Pu,v\rangle = \langle u,Pv\rangle$ f\"ur alle $u,v \in H$ gilt. 

\end{itemize}


\end{document}
